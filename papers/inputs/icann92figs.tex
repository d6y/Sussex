
\begin{figure}[hbt]
%
\begin{minipage}[b]{7.5cm} %  "b" to out the captions on the same line
\centerfig{ ../postscript/x01rt.ps }[40]
\caption{Plot of mean correct RT per multiplication table collapsed over
operand order for: median RT of ?? adults
\protect\cite[app.~D]{harlasso};
median RT (adjusted for naming time) of 6 adults
\protect\cite[table~A1]{millcogn};
mean RT for 20 networks trained on skewed frequencies; and, the same 20
networks after continued training on uniform frequencies (both networks
equally scalled).}
\label{rt01}
\end{minipage}   % no empty line here so the figures are side by side
%\hspace{0.3cm}
%
% \begin{minipage}[b]{7.5cm}
% \centerfig{ ../postscript/wtd01sum.ps }[30]
% \caption{Each large rectangle represents one
% hidden unit.  Within each rectangle, the size of the smaller rectangles
% represents the weighted sum to the hidden
% unit to a particular problem. Each large
% square mimics the multiplication table (top-left for \x00,
% and bottom-right for \x99), and the size of each small square represents
% the magnitude of the weigted sum.  Dark squares represent a inhibiting
% net input.}
% \label{wts01sum}
% \end{minipage}
%
% \begin{minipage}[b]{7.5cm}
% \centerfig{ ../postscript/freq3d.ps }[60]
% \caption{Relative frequency of multiplication problems, derived from
% frequencies found in 2nd- and 3rd-grade textbooks
% \protect\cite[p.~268]{siegmult}.  The far-right corner is \x00 and
% the near-left corner is \x99. Darker shading represents higher frequency.}
% \label{xfreq}
% \end{minipage}
\hspace{0.3cm}
%
\begin{minipage}[b]{7.5cm} %  "b" to out the captions on the same line
\centerfig{ ../postscript/p4x4_01.ps }[35]
\caption{Response of the output units over 40 time steps for the problem
\x44.  Output units representing products over 18 are not shown on this
graph. The ``don't know'' unit is represented by the ``?''.}
\label{act01plot}
\end{minipage}   % no empty line here so the figures are side by side
%
\end{figure}
