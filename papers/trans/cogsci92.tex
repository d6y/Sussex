\documentstyle[slidesonly,portrait]{seminar}


 % -- Title page
 % -- Overview
 % -- Reaction time
 % -- Types of slip
 % -- Spread of activation
 % -- The model
 % -- Experiments
 % -- Results (2-9)
 % -- Results (0-9)
 % -- Error Results
 % -- Development of errors
 % -- Other models
 % -- Future
 % -- Acknowledgements

\newcommand{\x}[2]{\mbox{$#1\times#2$}}

\begin{document}

% -- Title page -----------------------------------------------------------

\begin{slide*}

\heading{Memory for Multiplication Facts}

\begin{center}
\changefont{helvetica}\large
Richard Dallaway\\
\normalsize
richardd@cogs.susx.ac.uk\bigskip\\

University of Sussex\\
School of Cognitive \& Computing Sciences\\
Falmer, Brighton BN1 9QH, UK.\bigskip\\

\vspace{1cm}

\pssem{sussex.ps}{4cm}

\end{center}
\end{slide*}

% -- Overview -------------------------------------------------------------
\begin{slide*}
\heading{Overview}
\changefont{helvetica}\vspace{1 true cm}
\begin{center}
\begin{minipage}{5cm}
\begin{itemize}
\item Human empirical
\item Why bother?
\item Description of the model
\item Results
\item Analysis
\item Other PDP models
\item Future work
\end{itemize}
\end{minipage}
\end{center}

\end{slide*}

% -- Reaction time --------------------------------------------------------

\begin{slide*}
\heading{Reaction Time}

\vspace{0.5cm}
\pssem{../postscript/humanrt.ps}{4in}
\vspace{1cm}

\begin{itemize}
\item Problem-size effect: Small problems easier than large problems.

\item Exceptions to the rule:\\
\begin{itemize}
\item Five and nine times table
\item Tie problems $(\x22,\x33\ldots\x99)$
\end{itemize}

\end{itemize}


\end{slide*}


% -- Types of slip --------------------------------------------------------
\begin{slide*}
\heading{Types of Slip}

\begin{center}
\renewcommand{\arraystretch}{0.85}
\tabcolsep=3pt
\begin{tabular}{c|cccccccc}
$\times$&2&3&4&5&6&7&8&9\\
\hline
2&4& 6& 8& 10& 12& 14& 16& 18\\
3&6& 9& 12& 15& 18& 21& 24& 27\\
4&8& 12& 16& 20& 24& 28& 32& 36\\
5&10& 15& 20& 25& 30& 35& 40& 45\\
6&12& 18& 24& 30& 36& 42& 48& 54\\
7&14& 21& 28& 35& 42& 49& 56& 63\\
8&16& 24& 32& 40& 48& 56& 64& 72\\
9&18& 27& 36& 45& 54& 63& 72& 81\\
\end{tabular}

\newcommand{\vgap}{\vspace{4mm}}
\vgap
\begin{minipage}[t]{7.5 true cm}\raggedright
\parindent=0in
{\bf1. Operand}\\
For \x a b the error will be in the $a$ or $b$
table.\vgap

{\bf3. Frequent Products}\\
Ie: 12 16 18 24 36\vgap

{\bf5. Operation}\\
Eg: $3\times3=6$
\end{minipage}\hspace{1 true cm}%
\begin{minipage}[t]{7.5 true cm}
{\bf2. Close Operand}\\
Error will be: $(a \pm2)\times b$ or
$a\times (b\pm2)$\vgap

{\bf4. Table}\\
Eg: $4\times5=18$\vgap

{\bf6. Non-table}\\
Eg: $2\times3=5$
\end{minipage}
\end{center}

\end{slide*}

% -- Spread of activation -------------------------------------------------

\begin{slide*}
\heading{Measuring Response Time}

\newcommand{\net}{\mbox{net}}
$$
\net_i (t) = k \sum\limits_j w_{ij}a_j + (1-k) \net_i (t-1)
$$

%\pssem{../postscript/casfn.ps}{3.5in}
\pssem{../postscript/cascade.ps}{5.5in}


\end{slide*}
% -- The model ------------------------------------------------------------

\begin{slide*}

\heading{Architecture}
\vspace{2mm}
\pssem{../postscript/xnet.ps}{6.5in}

\heading{Input Representation}

\pssem{../postscript/inrep.ps}{4in}

\end{slide*}

% -- Experiments ----------------------------------------------------------

\begin{slide*}

\heading{Experiments}

\begin{itemize}

\item Training

\begin{itemize}
\item Skew in problem frequency\\
\ldots but for adults?

\begin{enumerate}
\item Skewed networks
\item Equalized networks
\end{enumerate}

\item Twenty different networks\\
(different initial weights)
\end{itemize}

\item Recall

\begin{itemize}
\item Random thresholds, 0.4 -- 0.9
\item Each problem presented 50 times
\end{itemize}

\end{itemize}
\end{slide*}


% -- Results (2-9) --------------------------------------------------------
\begin{slide*}
\heading{RT Results}

\pssem{../postscript/xrt.ps}{6in}
\end{slide*}

% -- Results (0-9) --------------------------------------------------------
\begin{slide*}
\heading{RT Results 0--9}

\pssem{../postscript/x01rt.ps}{6in}
\end{slide*}


% -- Error Results --------------------------------------------------------

\begin{slide*}
\heading{Error Results}

\begin{tabular}{lll}
I & can't & be bothered
\end{tabular}


\end{slide*}


% -- Development of errors

\begin{slide*}

\heading{The Development of Errors}

\pssem{../postscript/dev.ps}{6in}

\end{slide*}

% -- Other models ---------------------------------------------------------

\begin{slide*}

\heading{Other PDP Models}
\begin{itemize}
\item Graham's backpropagation experiments [1990]
\begin{itemize}
\item Many simulations
\item Detailed analysis
\item No explicit reaction mechanism
\end{itemize}

\item BSB: Brown University [1989]
\begin{tabular}{ll}
J.~A.~Anderson&Sue Viscuso\\
Kathy Spoehr&Dave Bennett
\end{tabular}

\begin{itemize}
\item RT based on settling time
\item Net couldn't learn all facts
\end{itemize}

\item MATHNET: Johns Hopkins [1991]
\begin{tabular}{ll}
Mike McCloskey&Grethe Lindemann
\end{tabular}
\begin{itemize}
\item Mean field theory networks
\item RT based on settling time
\item Nondeterministic errors by speeded settling
\end{itemize}

\end{itemize}
\end{slide*}

% -- Future ---------------------------------------------------------------

\begin{slide*}

\heading{Future Work}

\begin{itemize}
\item A rule for $\x0N=0$ ?
\item Brain damage studies
\item Mechanism for non-table errors
\item Prediction (10, 11, 12 tables)
\item Addition and multiplication
\item Other strategies (counting)
\item Verification tasks
\item Isolate important variables \& features
\item Multicolumn addition and multiplication
\end{itemize}

\end{slide*}


% -- Acknowledgements ----------------------------------------------------

\begin{slide*}

\begin{flushleft}
\changefont{helvetica}
\large

Thanks to:
\begin{center}
Prof.~Harry Barrow\\
Dr David Young
\end{center}

\vspace{1 true cm}

Funded by:

\begin{center}
The Science and Engineering\\ Research Council, UK\\
{\normalsize\em in conjunction with}\\
Integral Solutions Limited\\
\end{center}
\end{flushleft}
\end{slide*}

%%%%%%%%%%%%%%%%%%%%%%%%%%%%%%%%%%%%%%%%%%%%%%%%%%%%%%%%%%%%%%%%%%%%%%%%%%%
\end{document}
