% EDIN_MARCH93.TEX  (VanLehn's Wager)
% Talk presented to Edinburgh University PDP group
% Wednesday March 17th 1993
%    Requires the following postscript files:
%    eye.ps
\documentstyle[slidesonly,portrait,arithprob,alltt,bitem]{seminar}
\newcommand{\x}[2]{\mbox{$#1\times#2$}}
\newcommand{\anarrow}{\(\Rightarrow\)}
\begin{document}

%%%%%%%%%%%%%%%%%%%%%%%%%%%%%%%%%%%%%%%%%%%%%%%%%%%%%%%%%%%%%%%%%%%
% \begin{slide*}
% \heading{Overview}
% \begin{itemize}
% \item Aritmetic bugs
% \item Production systems
% \item Why bother with PDP?
% \item PDP account of bugs
% \item Conclusion
% \end{itemize}
% \end{slide*}

%%%%%%%%%%%%%%%%%%%%%%%%%%%%%%%%%%%%%%%%%%%%%%%%%%%%%%%%%%%%%%%%%%%

\begin{slide*}

\heading{Errors in children's arithmetic}

\begin{center}

\begin{arithprob}{llll}
&$5_{\ }$&$2_{\ }$&$4_{\ }$\\
$\times$&$7_{\ }$&$3_{\ }$&$1_{\ }$\\
\cline{1-4}$3_{\ }$&$5_{\ }$&$6_{\ }$&$4_{\ }$\\
\end{arithprob}


\begin{arithprob}{llll}
&$3_{\ }$&$4_{\ }$&$6_{\ }$\\
$\times$&&$2_{\ }$&$8_{\ }$\\
\cline{1-4}$2_{\ }$&$7_{3}$&$6_{4}$&$8_{\ }$\\
\end{arithprob}


\begin{arithprob}{ll}
&$4_{\ }$\\
$\times$&$2_{\ }$\\
\cline{1-2}$8_{\ }$&$8_{\ }$\\
\end{arithprob}


\begin{arithprob}{lll}
&$2_{\ }$&$4_{\ }$\\
$\times$&$3_{\ }$&$1_{\ }$\\
\cline{1-3}&$8_{\ }$&$4_{\ }$\\
\end{arithprob}


\end{center}

\end{slide*}

%%%%%%%%%%%%%%%%%%%%%%%%%%%%%%%%%%%%%%%%%%%%%%%%%%%%%%%%%%%%%%%%%%%
\begin{slide*}
\heading{Production system for multiplication}\bigskip

\scriptsize
\begin{alltt}
 INTO: [processmult]          \anarrow readintandb();
 SM:   [t ?t] [b ?b] [c ?c]   \anarrow do_calc();
 NX:   [next_top]             \anarrow [processmult] shift_top_left();
 WM:   [result ?u] [carry ?c] \anarrow writedown(); [next_top]
 CC:   [no_more_top]          \anarrow chkcy(); [chkbot] [addzero]
 CB:   [checkbottom]          \anarrow check_bottom();
 FI:   [none_left]            \anarrow [stop]\smallskip
 NB:   [no_more]              \anarrow endmult(); [startadd]
 CO:   [startadd]             \anarrow readincolumn();
 DA:   [column ?len ?dig]     \anarrow do_add();
 ML:   [next_left]            \anarrow [startadd] moveleft();
 WA:   [u ?u] [c ?c]          \anarrow writeadd(); [next_left]
 CA:   [no_more_digits]       \anarrow checkadd();
 AZ:   [addzero]              \anarrow add_zero();
\end{alltt}

\begin{alltt}
[ [3 4] [2 2] result=[6 8] topdigit=1 botdigit=2 carry=0]
\end{alltt}

\end{slide*}
%%%%%%%%%%%%%%%%%%%%%%%%%%%%%%%%%%%%%%%%%%%%%%%%%%%%%%%%%%%%%%%%%%%
\begin{slide*}
\heading{Repair theory and Sierra theory}

\bitem{Faulty rules lead to impasses}

{\small ``When
a constraint or precondition gets violated the student, unlike a
typical computer program, is not apt to just quit'' (Brown \& VanLehn,
1980).}

\bitem{Impasses need to be repaired}

\medskip\begin{tabular}{l@{\hspace{5mm}}l}
{\em Impasses}&{\em Repairs}\smallskip\\
Decision&Barge-on\\
Reference&No-op\\
Primitive&Back-up\\
\end{tabular}\medskip

\bitem{Rule and patterns are induced}


\end{slide*}
%%%%%%%%%%%%%%%%%%%%%%%%%%%%%%%%%%%%%%%%%%%%%%%%%%%%%%%%%%%%%%%%%%%

\begin{slide*}
\heading{Always-borrows-left}\medskip

\hspace{1cm}
\begin{arithprob}{llll}
&$3_{\ }$&$6_{\ }$&$5_{\ }$\\
$-$&$1_{\ }$&$0_{\ }$&$9_{\ }$\\
\cline{1-4}&$1_{\ }$&$6_{\ }$&$6_{\ }$\\
\end{arithprob}
\hspace{1.4cm}
\begin{arithprob}{lll}
&$6_{\ }$&$5_{\ }$\\
$-$&$1_{\ }$&$9_{\ }$\\
\cline{1-3}&$4_{\ }$&$6_{\ }$\\
\end{arithprob}

\noindent borrow-column = left-most and left-adjacent

\begin{itemize}
\item Skewed curriculum
\item Input language
\item Learns most specific rule
\end{itemize}
\end{slide*}

%%%%%%%%%%%%%%%%%%%%%%%%%%%%%%%%%%%%%%%%%%%%%%%%%%%%%%%%%%%%%%%%%%%
\begin{slide*}

\heading{Why connectionism?}

``If one had to choose between production systems and
connection systems as a representation language for knowledge
about subtraction, then production systems seem much more
plausible.  Indeed it is not easy to see how a connection system could
possibly generate the kind of extended, sequential problem-solving
behaviour that characterizes students solving subtraction problems''
(Kurt VanLehn, 1990).

``When people reach an impasse\ldots they do not simply halt and
issue and error message, as a computer would'' (VanLehn, 1990).

\bigskip
\centerline{\em Nor do neural nets}.

\end{slide*}


%%%%%%%%%%%%%%%%%%%%%%%%%%%%%%%%%%%%%%%%%%%%%%%%%%%%%%%%%%%%%%%%%%%
\begin{slide*}

\heading{What we know about arithmetic}\bigskip

\begin{itemize}
\item Children don't understand it
\item It takes time
\item Problems vary in length
\item Your eyes move
\item You don't always mark the page
\item It's learned in steps
\item Learning is by examples (not verbal recipes)
\item There are bugs and slips
\item Bugs mostly occur during testing
\end{itemize}

\bigskip
\heading{What we don't know about arithmetic}\bigskip

\begin{itemize}
\item What is read
\item What is done
\item How we navigate problems
\end{itemize}


\end{slide*}
%%%%%%%%%%%%%%%%%%%%%%%%%%%%%%%%%%%%%%%%%%%%%%%%%%%%%%%%%%%%%%%%%%%
\begin{slide*}
\heading{Real symbol processing}

\noindent {\em Schemata and sequential thought processes in PDP models.}
Rumelhart, Smolensky, McClelland and Hinton (PDP2, chapter 14).
\medskip

\noindent We are good at\ldots
\begin{itemize}
\item Pattern matching
\item Modelling our environment
\item Manipulating our environment
\end{itemize}

\noindent ``\ldots we succeed in solving logical problems\ldots by making
the problems we wish to solve conform to problems we are good at solving''.

\noindent I.e., turn long multiplication into lots of small input-output
associations.

\begin{arithprob}{lll}
$3_{\ }$&$4_{\ }$&$3_{\ }$\\
$8_{\ }$&$2_{\ }$&$2_{\ }$\\
\cline{1-3}&&\\
\end{arithprob}
\hspace{3cm}
\begin{arithprob}{lll}
$3_{\ }$&$4_{\ }$&$3_{\ }$\\
$8_{\ }$&$2_{\ }$&$2_{\ }$\\
\cline{1-3}&&$6_{\ }$\\
\end{arithprob}

\end{slide*}
%%%%%%%%%%%%%%%%%%%%%%%%%%%%%%%%%%%%%%%%%%%%%%%%%%%%%%%%%%%%%%%%%%%



%%%%%%%%%%%%%%%%%%%%%%%%%%%%%%%%%%%%%%%%%%%%%%%%%%%%%%%%%%%%%%%%%%%
\begin{slide*}
\heading{Focus of attention}\bigskip

\centerline{\pssem{../postscript/eye.ps}{4in}}
\bigskip

\centerline{\pssem{../postscript/longinrep.ps}{6in}}

\end{slide*}
%%%%%%%%%%%%%%%%%%%%%%%%%%%%%%%%%%%%%%%%%%%%%%%%%%%%%%%%%%%%%%%%%%%


%%%%%%%%%%%%%%%%%%%%%%%%%%%%%%%%%%%%%%%%%%%%%%%%%%%%%%%%%%%%%%%%%%%
\begin{slide*}
\heading{Operations}

\begin{center}\small
\begin{tabular}{ll}
Movement                 &  Registers\\\hline
\verb|top_next_column|   &  \verb|push_mark|\\
\verb|jump_answer_space| &  \verb|zero_accumulator|\\
\verb|jump_top_row|      &  \verb|next_answer_row|\\
\verb|left|              &  \verb|next_bottom_column|\\
\verb|right|             &  \verb|inc_answer_column|\\
\verb|up|                &  \verb|inc_top_column|\\
\verb|down|              &  \verb|add_start_position|\\
\verb|read_carry|        &  \verb|x_start_position|\bigskip\\
Writing                  &  Special actions\\\hline
\verb|write_units|       &  \verb|add_mark_to_accumulator|\\
\verb|write_tens|        &  \verb|compute_product|\\
\verb|mark_zero|         &  \verb|draw_rule|\\
\verb|mark_carry|        &  \verb|done|\\
\end{tabular}
\end{center}
\end{slide*}


%%%%%%%%%%%%%%%%%%%%%%%%%%%%%%%%%%%%%%%%%%%%%%%%%%%%%%%%%%%%%%%%%%%
\begin{slide*}
\heading{Architecture}\bigskip

\centerline{\pssem{../postscript/longxnet2.ps}{4.8in}}

\begin{center}
\footnotesize
\begin{arithprob}{ll}
&2\\
$\times$&3\\
\cline{1-2}
&6
\end{arithprob}
\end{center}


\begin{center}
\footnotesize
\begin{tabular}{ccll}
Task&Cell&Output&Comments\\\hline
$\times$&&\verb|x_start_pos|&Look at `3'\\
$\times$&Number&\verb|push_mark|&Remember the `3'\\
$\times$&Number&\verb|jump_top_row|&Move to the `2'\\
$\times$&Number&\verb|compute_product|&Multiply digits\\
$\times$&Number&\verb|jump_answer|&Move down\\
$\times$&Space&\verb|write_units|&Write `6'\\
$\times$&Space&\verb|done|&
\end{tabular}
\end{center}

\end{slide*}

%%%%%%%%%%%%%%%%%%%%%%%%%%%%%%%%%%%%%%%%%%%%%%%%%%%%%%%%%%%%%%%%%%%

\begin{slide*}
\heading{BPTT}\bigskip

\centerline{\pssem{../postscript/bptt.ps}{9cm}}

\end{slide*}

%%%%%%%%%%%%%%%%%%%%%%%%%%%%%%%%%%%%%%%%%%%%%%%%%%%%%%%%%%%%%%%%%%%

\begin{slide*}
\heading{Training}\medskip

\begin{enumerate}
\item One column addition, sum < 10.
\item Addition of three rows.
\item Two digit numbers, no carrying.
\item Addition with gaps.
\item Two digits with carry.
\item Three column addition, without carry.
\item Three column with carry.
\end{enumerate}

\noindent $1+1$, $1+1+1$, $11+11$\ldots \x11, \x{2}{5}, \x{11}{1},
\x{111}{1}, \x{12}{5}, \x{12}{9},
\x{1}{11}, \x{11}{11}, \x{1}{111}, \x{12}{15},
\x{12}{19}, \ldots \x{12}{99}, \x{111}{111}.

\end{slide*}
%%%%%%%%%%%%%%%%%%%%%%%%%%%%%%%%%%%%%%%%%%%%%%%%%%%%%%%%%%%%%%%%%%%
\begin{slide*}

\heading{Errors in network's arithmetic}

``\ldots many bugs are caused by testing beyond training\ldots'' (VanLehn
1990).

\footnotesize

\begin{arithprob}{llllll}
&&&$1_{}$&$1_{}$&$1_{}$\\
&&$\times$&&$1_{}$&$1_{}$\\
\cline{3-6}
&&&&$1_{}$&$1_{}$\\
&&$+$&&&$1_{}$\\
\cline{3-6}
&&&&$1_{}$&$2_{}$\\
\end{arithprob}
\hspace{1.5cm}
\begin{arithprob}{lll}
&$1_{\ }$&$2_{\ }$\\
$\times$&$1_{\ }$&$5_{\ }$\\
\cline{1-3}&$6_{1}$&$0_{\ }$\\
\end{arithprob}

\begin{arithprob}{lll}
&$1_{\ }$&$2_{\ }$\\
$\times$&$5_{\ }$&$0_{\ }$\\
\cline{1-3}&$5_{\ }$&$0_{\ }$\\
\end{arithprob}
\hspace{1.5cm}
\begin{arithprob}{lllll}
&&$0_{}$&$1_{}$&$2_{}$\\
&$\times$&$0_{}$&$9_{}$&$0_{}$\\
\cline{2-5}
&&&$0_{}$&$0_{}$\\
$+$&$0_{1}$&$0_{1}$&$8_{}$&$0_{}$\\
\cline{1-5}
&$0_{}$&&&\\
\end{arithprob}


\begin{arithprob}{lllll}
&&&$5_{}$&$9_{}$\\
&&$\times$&$1_{}$&$2_{}$\\
\cline{3-5}
&&$1_{}$&$1_{1}$&$8_{}$\\
$+$&&$5_{}$&$9_{}$&$0_{}$\\
\cline{1-5}
&&$6_{}$&$0_{}$&$8_{}$\\
\end{arithprob}
\hspace{1.5cm}
\begin{arithprob}{lllll}
&&&$1_{\ }$&$2_{\ }$\\
&&$\times$&$9_{\ }$&$0_{\ }$\\
\cline{3-5}&&$0_{\ }$&$0_{\ }$&$0_{\ }$\\
&$+$&$0_{1}$&$8_{\ }$&$0_{\ }$\\
\cline{2-5}&$0_{\ }$&$0_{\ }$&$8_{\ }$&$0_{\ }$\\
\end{arithprob}

\end{slide*}
%%%%%%%%%%%%%%%%%%%%%%%%%%%%%%%%%%%%%%%%%%%%%%%%%%%%%%%%%%%%%%%%%%%

%%%%%%%%%%%%%%%%%%%%%%%%%%%%%%%%%%%%%%%%%%%%%%%%%%%%%%%%%%%%%%%%%%%
\begin{slide*}

\heading{Processing trajectories}\medskip

\centerline{\pssem{../postscript/threeDtraj.ps}{8.5cm}}

\begin{itemize}
\item Operation deletion
\item Operation duplication
\item Operation substitution
\item Extra loops
\item Premature quits
\item Different ``repairs'' during learning
\end{itemize}

\end{slide*}
%%%%%%%%%%%%%%%%%%%%%%%%%%%%%%%%%%%%%%%%%%%%%%%%%%%%%%%%%%%%%%%%%%%



%%%%%%%%%%%%%%%%%%%%%%%%%%%%%%%%%%%%%%%%%%%%%%%%%%%%%%%%%%%%%%%%%%%
\begin{slide*}
\heading{Activation into action}\bigskip

\small
\noindent Example: network trained to \x{11}{1} and tested on \x{12}{56}.

\begin{center}\footnotesize
\begin{arithprob}{lll}
&$1_{\ }$&$2_{\ }$\\
$\times$&$5_{\ }$&$6_{\ }$\\
\cline{1-3}&$8_{1}$&$2_{\ }$\\
&&$6_{\ }$\\
\end{arithprob}
\end{center}

\centerline{\pssem{../postscript/twoDtraj.ps}{14cm}}

\end{slide*}

%%%%%%%%%%%%%%%%%%%%%%%%%%%%%%%%%%%%%%%%%%%%%%%%%%%%%%%%%%%%%%%%%%%
\begin{slide*}

\heading{Errors during learning}\bigskip

\small

Recording errors made when learning between \x{11}{1} and \x{111}{1}.
Testing net every 1500 epochs.\bigskip

\centerline{\pssem{../postscript/lesline.ps}{8cm}}\medskip

\begin{arithprob}{llll}
&$1_{\ }$&$1_{\ }$&$1_{\ }$\\
$\times$&$1_{\ }$&&$1_{\ }$\\
\cline{1-4}&&$1_{\ }$&$1_{\ }$\\
\end{arithprob}
\hspace{1.5cm}
\begin{tabular}{lll}
&{\em Should}&{\em Did}\\
14.&IAC&ITC\\
15.&ITC&JTR\\
16.&JTR&MUL\\
17.&MUL&DWN\\
18.&JAS&RDC\\
19.&RDC&UNI\\
20.&UNI&DONE\\
21.&DONE&\\
\end{tabular}

\end{slide*}

\begin{slide*}
\footnotesize

\tabcolsep=1pt
\begin{tabular}{lll}
14.&IAC&IAC\\
15.&ITC&JTR\\
16.&JTR&MUL\\
17.&MUL&DWN\\
18.&JAS&RDC\\
19.&RDC&RDC\\
20.&UNI&UNI\\
21.&DONE&ITC\\
22.&    &JTR\\
23.&    &MUL\\
24.&    &DWN\\
25.&    &RDC\\
26.&    &RDC\\
27.&    &UNI\\
28.&    &DONE\\
\end{tabular}
\hspace{1.5cm}
\begin{tabular}{lll}
14.&IAC&IAC\\
15.&ITC&JTR\\
16.&JTR&MUL\\
17.&MUL&DWN\\
18.&JAS&RDC\\
19.&RDC&RDC\\
20.&UNI&UNI\\
21.&DONE&RDC\\
22.&    &JTR\\
23.&    &JTR\\
24.&    &DWN\\
25.&    &ADD\\
26.&    &JAS\\
27.&    &RDC\\
28.&    &UNI\\
29.&    &DONE\\
\end{tabular}

\begin{arithprob}{llll}
&$1_{\ }$&$1_{\ }$&$1_{\ }$\\
$\times$&&&$1_{\ }$\\
\cline{1-4}&&$1_{\ }$&$1_{\ }$\\
\end{arithprob}
\hspace{1.5cm}
\begin{arithprob}{llll}
&$1_{\ }$&$1_{\ }$&$1_{\ }$\\
$\times$&&$1_{\ }$&$1_{\ }$\\
\cline{1-4}&$1_{\ }$&$1_{\ }$&$1_{\ }$\\
\end{arithprob}

\end{slide*}
%%%%%%%%%%%%%%%%%%%%%%%%%%%%%%%%%%%%%%%%%
\begin{slide*}
\footnotesize

\begin{arithprob}{llll}
&$1_{\ }$&$1_{\ }$&$1_{\ }$\\
$\times$&$1_{\ }$&&$1_{\ }$\\
\cline{1-4}&&$1_{\ }$&$1_{\ }$\\
\end{arithprob}
\begin{tabular}{lll}
14.&IAC&ITC\\
15.&ITC&JTR\\
16.&JTR&MUL\\
17.&MUL&JAS\\
18.&JAS&JAS\\
19.&RDC&RDC\\
20.&UNI&UNI\\
21.&DONE&UNI\\
22.&    &TNC\\
23.&    &JTR\\
24.&    &MUL\\
25.&    &DWN\\
26.&    &RDC\\
27.&    &RDC\\
28.&    &UNI\\
29.&    &DONE\\
\end{tabular}

\begin{arithprob}{llll}
&$1_{\ }$&$1_{\ }$&$1_{\ }$\\
$\times$&$1_{\ }$&&$1_{\ }$\\
\cline{1-4}&&$2_{\ }$&$1_{\ }$\\
\end{arithprob}


\end{slide*}

%%%%%%%%%%%%%%%%%%%%%%%%%%%%%%%%%%%%%%%%%%%%%%%%%%%%%%%%%%%%%%%%%%%
%%%%%%%%%%%%%%%%%%%%%%%%%%%%%%%%%%%%%%%%%%%%%%%%%%%%%%%%%%%%%%%%%%%
%%%%%%%%%%%%%%%%%%%%%%%%%%%%%%%%%%%%%%%%%%%%%%%%%%%%%%%%%%%%%%%%%%%
%%%%%%%%%%%%%%%%%%%%%%%%%%%%%%%%%%%%%%%%%%%%%%%%%%%%%%%%%%%%%%%%%%%
%%%%%%%%%%%%%%%%%%%%%%%%%%%%%%%%%%%%%%%%%%%%%%%%%%%%%%%%%%%%%%%%%%%
\begin{slide*}

\heading{Developmental trajectories}

\noindent Learning is\ldots
\begin{itemize}
\item Getting the right state transitions
\item Distinguishing between states
\end{itemize}

\noindent Bugs are\ldots
\begin{itemize}
\item Incorrect transitions\ldots
\item due to overlapping states
\end{itemize}

\noindent Repairs are\ldots
\begin{itemize}
\item Blended trajectories
\item Thanks to generalization, and
\item Similarity based processing
\end{itemize}

\medskip
Impasses don't occur

\end{slide*}

%%%%%%%%%%%%%%%%%%%%%%%%%%%%%%%%%%%%%%%%%%%%%%%%%%%%%%%%%%%%%%%%%%%


%%%%%%%%%%%%%%%%%%%%%%%%%%%%%%%%%%%%%%%%%%%%%%%%%%%%%%%%%%%%%%%%%%%
\begin{slide*}

\heading{Conclusions}\bigskip

\begin{itemize}
\item Repair process is inherent in the PDP architecture
\item Implementation of a production system?
\item Another micro-cognition account?
\item What's the psychological role of impasses and repairs?
\item Hybrid 1: build a PS with soft pattern matcher
\item Hybrid 2: try to do representation redescription
\end{itemize}

\end{slide*}

%%%%%%%%%%%%%%%%%%%%%%%%%%%%%%%%%%%%%%%%%%%%%%%%%%%%%%%%%%%%%%%%%%%

\end{document}
