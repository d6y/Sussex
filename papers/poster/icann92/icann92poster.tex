%% poster icann92poster.ps 22.5 30 | lpr
%% magnification 2.647059
\documentstyle[fontmix,psfig,multicol,spacebox]{article}

%\psdraft
%\spaceboxdraft

\psfigurepath{../postscript/}

\newcommand{\x}[2]{\mbox{$#1\times#2$}}
\def\net{\mbox{net}}

\columnseprule=0.4pt

% text area
\textwidth=8in
\textheight=11.5in
\oddsidemargin=-1in
\topmargin=-1in
\headheight=0pt
\headsep=0pt
\hoffset=5mm
\voffset=0.5in
\footheight=0pt
\footskip=0pt
\marginparwidth=0pt


\newcommand{\mysection}[1]{%\section{#1}}
    \bigskip\centerline{\changefont{\otherfont}\Large\bf#1}}
%    \begin{center}\changefont{\otherfont}\Large\bf#1\end{center}}

\newcommand{\figfile}[2]{\centerline{\psfig{file=#1.ps,#2}}}

\raggedright

\parskip=2mm
\parindent=0cm


\newcommand{\showtitle}{
    \begin{center}
       \changefont{\otherfont}\Huge
       {\bf The Speed and Slips of Mental Arithmetic}\smallskip\\
       \Large Richard Dallaway\\
       School of Cognitive \& Computing Sciences\\
       University of Sussex, Brighton UK.
    \end{center}
}

\begin{document}
\bibliographystyle{plain}

\pagestyle{empty}
\thispagestyle{empty}

\begin{multicols}{3}[\showtitle]


\mysection{Phenomena}


When asked to recall multiplication facts (such as, $\x34=12$), adults
exhibit a number of well documented behaviours.

In general, response time (RT) increases across the
multiplication table (\x23 is easier to answer than \x89).  Exceptions to
this ``problem size effect'' include the 5 times table
and ``tie'' problems (\x22, \x33, etc).\bigskip

\spacebox{5.5cm}{5.5cm}{RT graph}

Adults also make mistakes which can be classified into the following
(overlapping) categories:
\begin{enumerate}
\item {\em Operand errors.}  Most of the mistakes (92\%)
are multiples of one of the operands (so an incorrect answer to \x67 is
likely to be in the 6 or 7 times table).
\item {\em Close operand errors.} Operand errors are often close
in magnitude to the correct
answer.  For \x a b, the error will be correct
for $(a\pm2)\times b$ or $a\times (b\pm2)$.
\item {\em Table errors.} The erroneous product is correct for some
problem, but does not share any digits with the presented problem (e.g.,
\x65=16).
\item {\em Frequent products.}  Certain products that occur often in
the table also occur frequently as errors.
\end{enumerate}


In addition to the above observations, it is generally agreed that children
do not experience multiplication problems with equal frequency.  The
empirical findings on RT and errors were replicated by a
multilayered network trained on the problems \x00 to \x99 using
backpropagation.  A skew in problem frequency was found to be necessary to
achieve these results.


%%%%%%%%%%%%%%%%%%%%%%%%%%%%%%%%%%%%%%%%%%%%%%%%%%%%%%%%%%%%%%%%%%%%%%%%%%%

\mysection{Model}
\vspace*{3mm}

\figfile{x01net}{width=5.5cm,height=3.5cm}

During training, the presentation frequency of each pattern was skewed
according to the problem frequencies found in
school textbooks~[3].%\cite{siegmult}.


Although problems with small products do occur more frequently in
textbooks, there is no reason to believe this skew continues into
adulthood~[2]. Hence, after training on skewed data,
training continued using  problems of
equal frequency.  After training both the ``skewed'' and
``equalized'' networks correctly recalled all the patterns in the
training set.

\vspace*{1mm}

\spacebox{4.5cm}{4.5cm}{Problem freq}

%\cite{pdp3}

The ``cascade''
equation~[1] was used to simulate the
spread of activation in the network.  Each unit's activity builds
up over time:
\[ \net_i(t) = k \sum\limits_j w_{ij} a_j(t) + (1-k)\ \net_i(t-1)\]
\noindent where $k$ determines the rate with
which activation builds up (0.05).  The following is a plot of
output unit activity
for the problem \x44:

\vspace*{1mm}

\figfile{p4x4}{width=5.6cm}%,height=3.2cm}% {Activity plot}

The size of each square represents the activity of an output unit, with
time running left to right (70 cascade steps).

\mysection{Simulation Results}

For recall the network is started from a common ``don't know'' state for
all problems.  The network is trained to switch on a ``don't know unit''
when in this state. A random threshold is selected between 0.4 and 0.9, and
cascade processing proceeds until an output (product) unit exceeds the
threshold. Each of the 100 problems is presented 50 times, and the mean RT
(number of cascade steps) is recorded.

With a large threshold the correct product is reliably recalled, but
occasionally the threshold will be low enough to accept an incorrect
product. In the example run above, ``12'' would be a close operand error
for \x44.  Presumably the mild time pressure of the experimental situation
results in subjects ``lowering their thresholds''.  Incorrect answers
that occur during recall are placed into the various error categories.

\spacebox{4.2cm}{4.2cm}{Errors}

The resulting simulated RTs show some of the features of, and correlates
to, human data.  Also, the proportions of the various types of error match
those reported for adults.  This was achieved without the additional
assumptions suggested by other models.


\mysection{A rule for zeros?}

It has often been suggested that subjects use a rule $(\x0N=0)$ to answer
problems involving a zero.  The evidence often cited to support this is the
speed with which subjects answer zero problems.  The network described
here also answers zeros problems quickly, suggesting that a rule is not
required. However, because the zeros association is strong in the network,
zero occurs as an error in unrealistic situations.  Also, human
subjects display errors of the form $\x0N=N$, which is rarely observed in
the network.

These simulations, coupled with the observations of human zero errors,
indicates that it is the errors, and not the RT, which identifies that
there is something special about zero problems.

%\nocite{pdp3}\nocite{siegmult}\nocite{mcclmode}
%\bibliography{../wip/bib}
\end{multicols}
\end{document}
