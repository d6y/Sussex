\begin{abstract}

Arithmetic takes time.  Children need five or six years to
master the one hundred multiplication facts (\x00 to \x99), and it takes
adults approximately one second to recall an answer to a problem like \x78.
Multicolumn arithmetic (e.g., \x{45}{67}) requires a sequence of actions,
and children produce a host of systematic mistakes when solving such
problems. This thesis models the time course and mistakes of adults and
children solving arithmetic problems.  Two models are presented, both of
which are built from connectionist components.

First, a model of memory for multiplication facts is described. A
system is built to capture the response time and slips of adults recalling
two digit multiplication facts. The phenomenon is thought of as spreading
activation between problem nodes (such as ``7'' and ``8'') and product
nodes (``56'').  The model is a multilayer perceptron trained with
backpropagation, and McClelland's \citeyear{pdp3} cascade equations are
used to simulate the spread of activation.  The resulting reaction times
and errors are comparable to those reported for adults.  An analysis of
the system, together with variations in the experiments, suggest that
problem frequency and the ``coarseness'' of the input encoding have a
strong effect on the phenomena.  Preliminary results  from
damaging the network are compared to the arithmetic abilities of
brain-damaged subjects.

The second model is of children's errors in multicolumn multiplication.
Here the
aim is not to produce a detailed fit to the empirical observations of
errors, but to demonstrate how a connectionist system can model the
behaviour, and what advantages this brings. Previous production system
models are based on an {\em impasse-repair} process: when an child
encounters a problem an impasse is said to have occurred, which is then
repaired with general-purpose heuristics. The style of the connectionist
model moves away from this. A simple recurrent network is trained with
backpropagation through time to activate procedures which manipulate a
multiplication problem.  Training progresses through a curriculum of
problems, and the
system is tested on unseen problems.  Errors can occur during
testing, and these are compared to children's errors.  The system is
analysed in terms of hidden unit activation trajectories, and the errors
are characterized as ``capture errors''.  That is, during processing the
system may be attracted into a region of state space that produces an
incorrect response but corresponds to a similar arithmetic subprocedure.
The result is a {\em graded state machine}---a system with some of the
properties of finite state machines, but with the additional flexibility of
connectionist networks. The analysis shows that connectionist
representations can be structured in ways that are useful for modelling
procedural skills such as arithmetic. It is suggested that one of the
strengths of the model is its emphasis on development, rather than on
``snap-shot'' accounts. Notions such as ``impasse'' and ``repair'' are
discussed from a connectionist perspective.

\end{abstract}
