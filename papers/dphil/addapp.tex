\chapter[Addition bugs]{Addition Bugs}\label{c:aa}


In comparison to the research on subtraction, there have been few studies
which attempt to pin down multiplication and addition bugs. This lack of
data is an obvious problem, although as pointed out in
chapter~\ref{c:fsm}, it is not such a serious problem for this particular
project. Nevertheless, it is
important to know, at least roughly, what kinds
of bugs can occur and how frequent they are likely to be.


These appendices list the multiplication and addition bugs found in a
search of the literature.  All of the studies looked at have problems, and
the lists presented here do not remove the need for a large, modern study
of multiplication and addition.

\section{Sources}

\newbox\tcb
\setbox\tcb\hbox{\citeay{attiexpe}}

The first source of bugs comes from \citeA[also \unhbox\tcb]{attimicr}.
Attisha built a tutoring system for all four operations, and surveyed the
literature for bugs. Although a large number of bugs were listed, Attisha
failed to indicate their frequency or give examples with working marks.
Without working marks it is difficult to interpret the definitions of
certain bugs, namely
those which involve carrying.  Those definitions that
could be interpreted are included here, and working marks have been added
where it aids understanding of the bug.

One of the sources used by Attisha was the bug catalogue produced by
\citeA{coxanal}.  She studied the literature on arithmetic bugs from 1900
to 1973, and also conducted a study of bugs found in 564 subjects in grades
two to six.   The study demanded that the subjects be close to 100
per cent accurate on their number facts, and bugs
were only accepted if they
occurred at least three out of five times on a given type of problem.

Cox tested children on the four multicolumn tasks by having them complete
tests which were based around a number of levels.  For addition there were
eight levels, starting from addition of two digits to one digit without
renaming. Each level became increasingly difficult, up to addition of three
two-digit numbers with renaming. For multiplication there were ten levels.
Cox listed the bugs found at each level, and then over the levels produced
a categorization of the bugs.  As can be seen from table~\ref{f:coxcat},
Cox classified the bugs according to the kind of faulty knowledge that
caused the error. The meaning behind Cox's labels is obvious, except for
``concept'', which is the case when the child seem to be lacking a basic
understanding of the concept of multiplication or addition.

\begin{fancytable}
\begin{tabular}{lr@{\hspace{0.75in}}lr}
\multicolumn{2}{c}{Addition}&\multicolumn{2}{c}{Multiplication}\\
Renaming        & 23 &  Concept         & 19\\
Concept         & 17 &  Partial product & 13\\
Wrong operation & 6  & $\times$ after renaming & 10\\
Place value     & 5  & + after renaming & 7\\
& & Renaming        & 6\\
& & $\times$ by zero & 6\\
& & Wrong operation & 4\\
& & Reversal of digits & 2\\
\end{tabular}
\caption{Categories of bugs reported by \protect\citeA{coxanal},
and the number of different bugs that
fell under each category heading.}
\label{f:coxcat}
\end{fancytable}

Cox's analysis and method of testing makes it impossible to know exactly
how general or specific a particular bug is.  For example, for level 2
problems (adding a one digit number to a two digit number, renaming needed)
if the subject does not carry, is he or she exhibiting the bug \bug{nCAone}
or \bug{nCA}?  That is, if the subject was given a problem involving more
digits, would they fail to carry just in the first (ones) column, or in all
columns? In these appendices the most specific bugs are listed. Perhaps
this complication is why Cox lists the global bug frequencies based on bug
categories, rather than on specific bugs.

In her literature survey Cox missed the large arithmetic study undertaken
by \citeA{buswdiag}.  Using verbal protocols, Buswell described behaviours
for all four operations. The study lists the number of occurrences of
various ``habits''---consistent behaviours, but not necessarily behaviours
that could be classed as ``buggy''.  For example, one particular habit,
``added carried number last'', describes a subject who always added the
carry digit {\em after} adding up a column, rather than {\em before}
starting on the column. Buswell notes that occasionally the subject would
forget to add the carry, and this could be avoided if the subject added the
carry first (ibid., p.~160). This kind of behaviour does not fit with the
modern notion of a bug. However,
many of the habits described do appear to be bugs, and are included here.
Whereas Cox's descriptions may be over specific, Buswell suffers the
opposite problem. For example, Buswell describes the general bug
\bug{wrongop}, when other authors give more specific bugs like \bug{XforA}
or \bug{SforA}.

The final source of bugs comes from a small, unpublished undergraduate
project \cite{shar}. It is included here because it builds upon the work of
\citeA{younerro} and presents a production system model of multiplication,
as well as bug frequency information.   As such it is the
only study to date which can be easily compared to the computational
studies of subtraction.

\section{Notes on the catalogue entries}

%\input totals.tex
These two appendices list \Tbgs{} bugs. There are \Tx{} multiplication
bugs, of which \Txw{} do not have frequency information. For addition, the
total is \Tp{}, \Tpw{} without frequency data.

Each entry in the catalogue is laid out as follows. The bug name, given in
bold, is followed by a short description of the bug.  Most bugs have one or
more examples to clarify the description. The source of the bug is
indicated by showing the name of one of the authors mentioned in the
previous section.  If that author gave frequency information, it is shown
as a percentage of all the bugs taken from that author. Note that this will
not be a percentage of all
the bugs listed by that author: bugs were not included either
because they were not clearly described, or because they were not relevant
(e.g., number fact errors).  For addition, 116 bug occurrences were
used from Cox, and 484 from Buswell.  A total of 113 occurrences of
multiplication bugs were used from Cox, 512 from Buswell, and 76 from
Ainsworth.

The catalogue is listed in alphabetical order, but
table~\ref{f:freqlist} (on page~\pageref{f:freqlist}) lists
the most frequent bugs in order of
frequency---or an approximation to that given that many bugs have frequency
values from two or three authors.

Some of the bugs listed produce results that look identical to other bugs.
For example, the bugs \bug{dnRsum} and \bug{dnRP} both result in the
subject writing carry digits in the answer row. However, these bugs qualify
as separate bugs because they have been observed independently of each
other. That is, a subject can fail to rename a partial product, yet
correctly rename when adding the partial product.

\begin{table}[t]\small
\begin{rotate}\vbox{%
\begin{center}
\begin{sstabular}
\input /csuna/home/richardd/papers/dphil/freqlist.tex
\caption{%
The 28 most frequent addition and multiplication bugs. {\figfont
Key:} Values are percentages from three authors,
A=\protect\citeA{shar}, N=76;
B=\protect\citeA{buswdiag}, N=512 for multiplication, N=484 for addition;
C=\protect\citeA{coxanal}, N=113 for multiplication, N=116 for addition.}
\label{f:freqlist}
\end{sstabular}\end{center}}\end{rotate}
\end{table}



%It should be remembered that bugs can and do occur together in one
%subject.
%Cox noted a number of bugs that are really combinations of bugs.  For
%example, she describes a bug in which the subject subtracts rather than
%add, but subtracts the smaller number from the larger number.  In this
%list, the bug is classified under \bug{SforA}.

%Cox listed some bugs which appeared to be a combination of two
%other bugs.  For example, one bug given was a combination of the bug
%\bug{Rtowrongcolumn} with \bug{onetoomany} (or possibly \bug{CAtoN}). In
%this particular case, as the bug only occurred once, I incremented the
%frequency could for \bug{Rtowrongcolumn} {\em and\/} \bug{onetoomany} by a
%half.


The Buswell frequencies are based on the total frequencies made by 263
subjects, spread over grades 3 to 6 \citeyear[tables XXXV to~XXXVII,
pp.~136--139]{buswdiag}.  From the Ainsworth study, the frequencies are
summed over two sets of 10--11 year olds and one group of 8-9 year olds
\cite[table~2, p.~32]{shar}.

The model described in chapter~\ref{c:fsm} does not attempt to model
certain kinds of errors, and for this reason some space-saving liberties
have been taken in these appendices. In particular, pattern errors, like
\bug{nXZEn}, are only given one way round (i.e., \x0N=N
is not shown). In the Cox, Ainsworth and Buswell studies, they are given
both ways as they can occur independently.

\section*{Addition bugs}

\begin{singlespace}
\input /csuna/home/richardd/papers/dphil/addlist.tex
\end{singlespace}
