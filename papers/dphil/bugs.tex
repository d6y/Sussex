
% Insert BLURB at start of Empirical adequacy

\part[Multicolumn arithmetic]{Multicolumn Arithmetic}\label{c:mult}

\newcommand{\anarrow}{\(\Rightarrow\)}

\chapter[Symbolic accounts of arithmetic]{Symbolic Accounts of Arithmetic}
\label{c:bugs}

Different tasks lend themselves to different styles of representation
\cite{slomwhy}.  As a rule of thumb it makes sense to
use the most appropriate technology to model the phenomena of
interest. An example would be to use connectionism for low-level
processes (e.g., motor control), but switch to symbolic systems for
higher-level tasks, such as planning \cite{micro}.  Such hybrid views of
cognition are attractive \cite{thorstir,thorhybr,roseappr,hendneed}, but
how do we know which technology is most appropriate for a given task?

This chapter reviews the symbolic models that have been built to capture
the way children learn multicolumn arithmetic.  Solving problems
like 32$-$27 or \x{49}{12} involves a host of skills: not only do you
need to know the arithmetic facts, you also need to know how to
borrow and carry, which column to process next,
what to do when a number does not have a number below it
(as in 12+5), and so on.

It appears that students are following rules when solving multicolumn
problems.  Perhaps the main evidence for this comes from the observation
that children discover faulty rules (malrules) when learning arithmetic.
Hence, it seems appropriate to use something like a production system to
model multicolumn arithmetic. Indeed, to date the only systems
used to investigate arithmetic have been rule-based systems
\cite{buggy,younerro,repair,vanlfeli,mindbugs}.

This chapter first looks at the kinds of mistakes children make when
solving multiplication problems.  The style of symbolic modelling is
then described by briefly discussing a production system for
multiplication.  Section~\ref{s:yo} considers the \citeA{younerro}
approach to modelling children's errors, and
section~\ref{s:sierra} looks at the way \citeA{mindbugs} has modelled
the errors.   The conclusion is that the symbolic models offer a very
plausible interpretation of children's arithmetic.  Nevertheless,
section~\ref{s:whypdp} argues that there are good reasons for looking
into a connectionist account of the phenomena.  In particular, I aim to
show that connectionism may be an ``appropriate technology'' for this
domain, despite the phenomena's rule-like appearance.
Many of the assumptions
and ideas from the VanLehn and Young \& O'Shea accounts are taken on
board in chapter~\ref{c:fsm} which describes a connectionist model of
multicolumn arithmetic.

\section{Bug phenomena}\label{s:bugphen}

At school children are introduced to multicolumn arithmetic over a number
of years in lessons of ever-increasing difficulty.
Maths textbooks \cite{howemath} suggests the following problem ordering
for addition:

\ssadjust
\begin{enumerate}
\begin{singlespace}
\item One column addition, sum < 10.
\item Two digit numbers, no carrying.
\item Addition of three rows.
\item Addition with gaps.
\item Two digits with carry.
\item Three column addition, without carry.
\item Three column with carry.
\end{singlespace}
\end{enumerate}

There is a similar sequence for multiplication, and \citeA[p.~13]{mindbugs}
identifies one for subtraction.  By working through these stages students
build up a hierarchy of maths skills, enabling them to tackle the next
level of problems.  Interestingly, \citeA[p.~55]{resnpsyc} discuss studies
involving a ``deep end'' approach to arithmetic teaching,
in which students are just taught the
higher level skills.  The reasoning behind this idea is that the student
should be given meaningful problems, rather than many boring, relatively
simple problems.  Of course, students taught this way still end up learning
all the prerequisite skills.  However, the majority of students seem to
learn best when led through a hierarchy of skills, and this is the way
arithmetic is taught in most schools.

Each lesson typically begins with the teacher working an example, and then
the children solve similar problems on their own.  Textbooks work in much
the same way.  An example is worked, by printing a snapshot of the various
stages that need to be completed, and then there is a list of exercises.

There are at least two multiplication algorithms that are taught
in schools.  The one considered here involves building up the product of
a multiplier on one row. For example:

\begin{arithprobB}{lll}
&$1_{\ }$&$2_{\ }$\\
$\times$&$3_{\ }$&$4_{\ }$\\
\cline{1-3}&&$8_{\ }$
\end{arithprobB}
\hfil
\begin{arithprobB}{lll}
&$1_{\ }$&$2_{\ }$\\
$\times$&$3_{\ }$&$4_{\ }$\\
\cline{1-3}&$4_{\ }$&$8_{\ }$
\end{arithprobB}
\hfil
\begin{arithprobB}{lll}
&$1_{\ }$&$2_{\ }$\\
$\times$&$3_{\ }$&$4_{\ }$\\
\cline{1-3}&$4_{\ }$&$8_{\ }$\\
&&$0_{\ }$
\end{arithprobB}
\hfil
\begin{arithprobB}{lll}
&$1_{\ }$&$2_{\ }$\\
$\times$&$3_{\ }$&$4_{\ }$\\
\cline{1-3}&$4_{\ }$&$8_{\ }$\\
&$6_{\ }$&$0_{\ }$
\end{arithprobB}
\hfil
\begin{arithprobB}{lll}
&$1_{\ }$&$2_{\ }$\\
$\times$&$3_{\ }$&$4_{\ }$\\
\cline{1-3}&$4_{\ }$&$8_{\ }$\\
$3_{\ }$&$6_{\ }$&$0_{\ }$
\end{arithprobB}
\hfil
\begin{arithprobB}{lll}
&$1_{\ }$&$2_{\ }$\\
$\times$&$3_{\ }$&$4_{\ }$\\
\cline{1-3}&$4_{\ }$&$8_{\ }$\\
$3_{\ }$&$6_{\ }$&$0_{\ }$\\
\cline{1-3}$4_{1}$&$0_{\ }$&$8_{\ }$
\end{arithprobB}\skipafterprob


Studies of errors in arithmetic \cite{buggy,debuggy,repair}
have characterized
students as having buggy procedures---perturbations to the correct
procedure.  For example, one error for multiplication (\bug{firstZ})
begins like this:

\begin{arithprob}{llllll}
&&&$4_{\ }$&$3_{\ }$&$6_{\ }$\\
&&$\times$&&$5_{\ }$&$1_{\ }$\\
\cline{3-6}&&$4_{\ }$&$3_{\ }$&$6_{\ }$&$0_{\ }$\\
\end{arithprob}\skipafterprob

The student has incorrectly inserted zero into the first column.
Another error is \bug{XlikeA}:

\begin{arithprob}{lll}
&$3_{\ }$&$4_{\ }$\\
$\times$&$1_{\ }$&$2_{\ }$\\
\cline{1-3}&$3_{\ }$&$8_{\ }$\\
\end{arithprob}\skipafterprob

Here the student has multiplied each digit in the lower row
by the digit above it in the upper row.
These kinds of errors are quite different from the recall errors discussed
in previous chapters. Rather than slipping and incorrectly following the
correct arithmetic procedure, students are systematically using faulty
rules. \citeA{vanlbugs} reports that students exhibiting a particular bug
can reproduce that bug a week later, giving exactly the same mistakes,
digit for digit.  However, there are limits to this stability, and this is
discussed below.


\input totals.tex

Many different kinds of bugs have been described, some frequent, some rare.
The most specific bug catalogues previously published have been for
subtraction. I have decided to look at long multiplication and addition, so
appendix~\ref{c:aa}
and \ref{c:xa} lists \Tp{} addition and \Tx{} multiplication bugs found in
a trawl of the literature.
Details of any particular bug, such as \bug{firstZ}, can be found in these
appendices.  The descriptions use the terminology introduced in
figure~\ref{f:xterms}.
\begin{fancyfigure}
\centerline{\psfig{file=xterms.ps,height=5.5cm}}
\caption{Terminology used for multiplication problems.  For two
row addition, the problem numbers are referred to as the ``addends''.}
\label{f:xterms}
\end{fancyfigure}

There is a great deal of interpretation in classifying a given mistake as
being one bug rather than another.  For example, if the child makes the
mistake \x60=6 as part of a problem, is that the bug \bug{nXZEn}, or
\bug{copymultiplicand}, or even \bug{AinXpattern}?  Another problem is
deciding that a given error is systematic.  Different authors tend to take
different approaches. The appendices are the result of combining the bugs
listed by three authors, so at one level the problem of interpretation was
solved because the authors had already classified the bugs.  However, as
discussed in appendix~\ref{c:aa}, the interpretation problem then becomes
one of finding corresponding bugs between the authors.  The controversy
surrounding bug categorization is not pursued further here, and it is
assumed that at least some of the errors made by children are well
described by buggy rules.

A number of studies have been conducted to discover the frequency of bugs.
Of the 1147 seven- and eight-year-old children from VanLehn's Southbay
study, 33 per cent harboured bugs, and a total of 134 distinct bugs were
identified.  Individuals may have more than one bug, and their set of bugs
will change over time. \citeA{coxanal} found that around 56 per cent of the
564 grade 1--6 children in her study made systematic errors in addition,
multiplication and division. However, 10 percent were making ``random
errors'', and the remaining children were error-free or just slipping.
\citeA[p.~6]{vanlbugs} also found that 10 per cent of grade 3 children's
error could not be analysed in his study of subtraction.  Still, at most 90
per cent of children's behaviour can,
in principle, be explained by reference to a possibly faulty rule set.


\section{Models}\label{s:mmods}

The formalization of the child's rule set is discussed in this
section---including rule acquisition, operation and representation.
Starting with the account of arithmetic developed by
\citeA{younerro}, I hope to show how the phenomena can be neatly
modelled by production systems.  Repair theory \cite{repair} appeared at
about the same time as the Young \& O'Shea model, and has since gone
through many refinements to reach its current form as Sierra
theory \cite{mindbugs}.  Despite a number of criticisms,
Sierra theory is the best account we have of faulty rule acquisition.  The
reasons for this are described below.

\subsection{A production system for multiplication}

%see \citeay{andearch}, pp.~7--12 for an addition production system).

Young \& O'Shea produced a production system to model correct and buggy
subtraction. In the same style as the Young \& O'Shea system a
multiplication production system was built by Shaaron Ainsworth as
part of her MSc work at Sussex.
Table~\ref{f:xprod} lists the production rules for correct multiplication.
In what follows, many of the
implementation features of
this particular system are glossed over,
and it is assumed that the model works in much the same way as the
Young \& O'Shea system.

\begin{fancytable}
\begin{alltt}
      {\em Conditions}                 {\em Actions}\smallskip
 INTO: [processmult]          \anarrow readintandb();
 SM:   [t ?t] [b ?b] [c ?c]   \anarrow do_calc();
 NX:   [next_top]             \anarrow [processmult] shift_top_left();
 WM:   [result ?u] [carry ?c] \anarrow writedown(); [next_top]
 CC:   [no_more_top]          \anarrow checkcarry(); [checkbottom] [addzero]
 CB:   [checkbottom]          \anarrow check_bottom();
 FI:   [none_left]            \anarrow [stop]\smallskip
 NB:   [no_more]              \anarrow endmult(); [startadd]
 CO:   [startadd]             \anarrow readincolumn();
 DA:   [column ?len ?dig]     \anarrow do_add();
 ML:   [next_left]            \anarrow [startadd] moveleft();
 WA:   [u ?u] [c ?c]          \anarrow writeadd(); [next_left]
 CA:   [no_more_digits]       \anarrow checkadd();
 AZ:   [addzero]              \anarrow add_zero();
\end{alltt}
\caption{Production rules for correct multiplication.}
\label{f:xprod}
\end{fancytable}


As is usual of production systems the left-hand side of the productions
specify queries to be matched against a working memory.  The right-hand
side specify actions which can be either calls to procedures
(e.g., \verb|do_calc()|) or items to deposit in the working memory
(such as \verb|[processmult]|, the initial goal).
In working memory the partial solution is represented
by a list containing: the two digits being multiplied; the result of the
current row of multiplications; two numbers indicating which digits in the
multiplicand and multiplier are being considered; and, the number currently
being carried.  For example, while solving the problem \x{34}{22}, the
representation will be\ldots
\begin{alltt}
  [ [3 4] [2 2] result=[6 8] topdigit=1 botdigit=2 carry=0 ]
\end{alltt}
\ldots indicating that the digits just considered were 3 (\verb|topdigit|
is 1, the first number in \verb|[3 4]|) and 2 (the second number in
\verb|[2 2]|).  The result of the row is 68, with no carry.  In the Young
\& O'Shea model there was no explicit representation of \verb|topdigit| or
\verb|botdigit|, or their equivalents.  Their production system used
operations such as \verb|ShiftLeft| and \verb|NextColumn| to focus
attention without worrying about the underlying registers.  However,
registers are required, and it is useful to have them explicitly shown.

Using this representation
of the problem, and the \verb|topdigit| and
\verb|botdigit| registers, the answer can be gradually
built up in working memory. The procedure \verb|readintandb|,
for example, identifies the digits to be multiplied.
\verb|Shift_top_left|
updates the \verb|topdigit| and \verb|botdigit| registers to focus on the
next digits in the problem.  Eventually, \verb|shift_top_left| runs out of
digits and deposits \verb|[no_more_top]| in working memory.  This then
caused the productions system to enter into a different part of the
arithmetic routine, ready to process the next multiplier.

The details of this model are not of great interest here, but note the
style of processing.  The model captures multiplication as small skill
modules, which are pattern or event driven.  For example, certain actions
bring about the recognition of certain other facts about the problem, which
leads to other rules firing as happened with \verb|[no_more_top]|.
Central to the model is the use of registers, such as \verb|topdigit|, to
keep track of the system's position in the problem. Registers are also used
in the eye movement models of arithmetic proposed by \citeA{suppproc}, and
will be used in the connectionist model described in the next chapter.
Finally note that the skill has been described at a level equivalent to how
one might describe multiplication verbally.  Clearly there are other
processes going on which are not modelled by the production system,
such as eye movements, recall of arithmetic facts, physical action of
writing, and so on.  However, it seems that this level is adequate for
capturing arithmetic bugs.

\subsection{Modelling bugs---the Young \& O'Shea way}\label{s:yo}

Certain bugs can easily be modelled by removing a rule from the production
system. For example, removing the AZ rule (from figure~\ref{f:xprod})
produces the bug
\bug{noleadingZ}.  Other bugs require that rules in the rulebase are
replaced with faulty versions.  For example, \bug{Qafterfirstbottom} is
generated by replacing the rule CB with the following:
\begin{alltt}
    bugCB:   [checkbottom]   \anarrow   [no_more]
\end{alltt}

Young \& O'Shea modelled errors in subtraction in this way, by adding and
omitting various rules.  They discovered that a small number of changes
could cover a large number of observed errors.  As an example Young \&
O'Shea note (p.~163) that two rules (one correct, one faulty), in four
possible permutations (each either present or absent from the rulebase)
could account for 115 out of 124 errors found in their corpus, in addition
to the correct algorithm.

To account for all the observed errors, Young \& O'Shea included a number
of other faulty rules.
Clearly, it would be trivial to construct a
production system to account for a particular error, and such a production
system would be of no psychological use.
However, an attempt was made to constrain the
productions in various ways.
Young \&
O'Shea's aim was
to model a particular child with a particular production
system.
Different children will have different productions, and each child's set of
rules will change over time.  Given this constraint, it is still possible
to build a production
system ``\ldots whose conditions were
so complex, bizarre, and ad hoc that each subtraction problem was
effectively treated as a separate case\ldots'' \cite[p.~164]{younerro}.
This consideration applies to all cognitive models, and it is not possible
to exactly specify a set of constraints to avoid ad hoc models.  Young \&
O'Shea used three heuristics to guide their model building:
\begin{enumerate}
\item Adopt a particular style for the rules.
\item Avoid problem-specific symbols in the rules (i.e., numbers).
\item Minimize the number of changes between rulebases.
\end{enumerate}


Using these heuristics, Young \& O'Shea optimized a number of production
systems to fit the errors found in their data.  Starting with the correct
production system for subtraction, rules were changed to improve a score.
The score is the number of errors predicted by the system, minus the number
of false errors. False errors are not errors which have never been observed
(star bugs, discussed below), but are due to the fact that children are not
fully consistent in applying faulty rules. It may be that a child
exhibiting a particular error does not make the same mistake at every
opportunity. A production system following a rule will always make the
error. False errors are the mismatch between the child's and the system's
error performance. Initially, the correct system makes no false errors, and
predicts no bugs.  Young \& O'Shea mutated the rulebases and finished with
eight production sets which accounted for 160 of the children's errors,
missing 18 and falsely predicting another 32 \citeyear[table~3,
p.~166]{younerro}.


It is
interesting that arithmetic skills can be captured in this way, but
there is no account of where the rules (correct or otherwise) come from.
The modularity of productions means that learning can be thought of as the
accumulation of rules \cite{younprod,nechlear}, although no such account is
given. The Young \& O'Shea model provides snapshots of children at various
stages of development, but does not discuss the transition between stages
(acknowledged on p.~176).
There are ways to incorporate learning processes into
the Young \& O'Shea ``family'' of models.  \citeA{klahinfo} lists a number
of mechanisms, including condition generalization and discrimination, rule
composition and chunking,
proceduralization, and rule strengthening.  There is a gradual
nature to some of these mechanisms, such as rule strengthening, but not to
all of them.  At some stage new rules need to be added to the rule base
\cite[p.~170]{klahinfo}.
This is what is meant by the phrase ``snapshot account'': at one moment the
rule base contains a certain set of rules, and at the next is contains some
new ones.  The Young \& O'Shea model is an extreme example, but the
granularity of other models (including VanLehn's) means that they also
qualify as snapshot accounts of development. This style of
learning will be contrasted with the gradual learning performed by
connectionism later in the chapter.
Rather than discuss ways in which learning
could be incorporated into the Young \& O'Shea model, attention will be
focussed on VanLehn's learning system in section~\ref{s:induct}.

Inconsistent rule following is not accounted
for by the Young \& O'Shea model. The term ``bug migration'' \cite{repair}
refers to the phenomenon of children switching between sets of bugs.
Some bugs are persistent, lasting for years, while others come and go
over a short period to time---even during a single test.
\citeA{hennwhy} believes that there is
much more instability in children's performance than has been previously
suggested.  She argues (pp.~178--180) that a more accurate, and less
subjective means of bug diagnosis is required.  The \citeA{coxanal}
study defined a bug as an incorrect procedure observed on three out of
five problems.  This kind of definition leads to the three error
categories of slip, bug and undiagnosed.  \citeA{vanlbugs} uses a
different, more complicated method, designed to ``mimic the intuitions of
human diagnosticians'' (p.~15).  A student is said to have a particular
bug if the diagnosis makes ``enough'' true predictions, and at least
more true predictions than false predictions. ``Enough'' is defined by a
number of conditions, e.g., 75 per cent of predictions must be true
\cite<for further discussion on the problems of diagnosis,
see>{buggy,debuggy}. For Hennessy, any faulty procedure counts as a bug,
no matter how frequent it is, with the exception of slips.  This
definition clearly eliminates the ``undiagnosed'' category of errors,
but increases the instability associated with a given student.

A number of studies have attempted to pin-down the extend of bug
migration.
In a follow-up study, one year after her original study,
\citeA{coxanal} found that 25 per cent (16 of 64) students exhibited the
same bug as in the previous year,
and 17 per cent (11 of 64) were showing a different error
over all the arithmetic operations.
VanLehn \citeyear{vanlbugs,mindbugs}
found
that in the short term (over 2 days) only 9 per cent (3 of 32) students
produced the same errors, where as 38 per cent (12 of 32) exhibited bug
migration for subtraction only.  The different ages of the children,
different experimental procedures, and small sample sizes makes it
difficult to compare the Cox and VanLehn results on bug migration.

Despite the problems of diagnosis, it is apparent that bug migration is
a common phenomenon.  Migration is usually defined as changes occurring
``without intervening instruction'' \cite[p.~54]{mindbugs}, but this does
not exclude the possibility that some kind of learning is causing
migration.
However, the
Young \& O'Shea model does not address bug migration or
learning, and as such it fails to capture important aspects of children's
arithmetic.


\subsection{Modelling bugs---the VanLehn way}\label{s:sierra}

VanLehn's \citeyear{mindbugs} Sierra theory of learning arithmetic is
the most detailed account of multicolumn arithmetic to date.  Like Young
\& O'Shea, he
focuses
on long subtraction, but asserts that the principles should apply not
only across the other three arithmetic procedures, but also to other
domains. The account, a continuation of repair theory \cite{repair},
tackles two
problems of bug migration and learning. Three elements of his
theory stand out:
\begin{enumerate}
\item Children's mistakes are the result of various kinds of
{\em syntactic} changes to rules.  This not only includes the faulty
rules from the Young \& O'Shea account, but also run time changes.
\item Faulty rules are due to the {\em skew} in the school curriculum.
\item Learning is from {\em examples}, not verbal recipes.
\end{enumerate}


Several observations support these points. First, it seems that students
do not understand the operations they are performing: they are
symbol-pushing \cite[pp.~38--40]{mindbugs}.  There is now work
considering the role of the semantics of arithmetic
\cite{hennwhy,paynalge,ohlsfunc} but there are a number of reasons why
it seems appropriate to just model syntax. To support his
``ateleological assumption'', VanLehn points out that students who
understand the arithmetic procedure they are trying to perform should be
able to see if their procedure is wrong, and also be able to fix it.
Intuitively it is clear that students can learn procedures without the
slightest understanding of the underlying principles, and this point
appears uncontroversial \cite[p.~38]{mindbugs}.

%This doesn't seem to rule out the possibility that students have a
%false understanding of arithmetic, but considering such a proposal
%ignores Occam's razor: it turns out that children's arithmetic can be
%modelled without a consideration of their understanding of arithmetic.
%Intuitively it's clear that students can learn procedures without the
%slightest understanding of the underlying principles, and VanLehn
%assures us (p.~38) that the ateleological assumption is
%uncontroversial.


The work of \citeA{hughwhat} also suggests that children have trouble with
the syntax of arithmetic:
\begin{ssquote}
The problem is not that young children are completely lacking in their
number concepts\ldots Rather the problem is that they are encountering a
novel code, or representation system, which may be like a foreign language
to them (p.~209).
\end{ssquote}

In other words, school
arithmetic is, as \citeA{donachil} calls it, ``unembedded''.  This
is demonstrated by the following protocol from a
four year old reported by \citeA[p.~211]{hughwhat}:

\begin{sstabular}
\begin{tabular}{ll}\\
Adult: &How many is two and one more?\\
Child: &Four.\\
Adult: &Well, how many is two {\em lollipops} and one more?\\
Child: &Three.\\
Adult: &How many is two {\em elephants} and one more?\\
Child: &Three.\\
Adult: &How many is two {\em giraffes} and one more?\\
Child: &Three.\\
Adult: &So how many is {\em two} and one more?\\
Child: &(looks adult straight in the eye) Six.
\end{tabular}
\end{sstabular}
\spaceafterquote

These problems with syntax should come as no surprise given the way
arithmetic is taught. \citeA[p.~82]{vanlfeli} points out that arithmetic
texts are nothing like cookbooks or other kinds of manuals: they consist
mostly of worked examples and exercises.  In contrast to examples are
``explanations''---some
form of natural language information, perhaps given
by the teacher in class.
VanLehn argues \citeyear[pp.~96--103]{mindbugs} that arithmetic is
primarily learned from examples, and not explanation.  His justification
of this assumption first notes that AI has experienced a number of
problems in translating from natural language to programs.  However,
there has been much more success in learning from examples.  Second,
VanLehn supposes that if children learned mostly via
natural language, there should be language fragments in their bugs, such as
references to the ``tens place'' or the ``multiplicand''.
However, VanLehn found that
``\ldots 85 percent of all the observed bugs can be described with a
small set of visual/spatial features.  No bug requires linguistic
features for its description'' (p.~102).

So it seems that children learn the steps in arithmetic procedure by
example, without an understanding of the algorithm. But how do
they acquire these procedures?

\subsubsection{Learning by induction}\label{s:induct}

Like Young \& O'Shea, VanLehn proposes that children have buggy core
procedures.  However, unlike Young \& O'Shea, VanLehn shows how these
procedures can be induced
from the examples given in a lesson.  This section
describes the learning part of Sierra, the name of VanLehn's
model. In
general terms, there
is a learner which takes a lesson, $L_i$, and the current state of a
student, $P_{i-1}$, and returns a set of procedures that are consistent
with the examples in the lesson ${P_i^1, P_i^2,\ldots,P_i^k}$.
That is, the learner returns a
number of rule sets, some with bugs, and some without.  Each of the
procedures can be tested to see how its bugs compare to those of children.
Then another lesson is administered, and the learner
generates procedures that can solve harder problems.

Much of VanLehn's \citeyear{mindbugs} work is concerned with ways to
constrain the learner
to induce just those things that humans learn.  Hence, a number of
assumptions are laid out, including:
\begin{itemize}
\item New rules are assimilated with the old. That is, the rules of
$P_{i-1}$ are a subset of those of $P_i$.
\item One disjunct per lesson.  At most one subprocedure (a new branch
point) can be introduced to a rule in a lesson.
\item The most specific patterns are induced.
\end{itemize}

Using a rule representation it is possible to parse arithmetic examples
(see figure~\ref{f:parse}).  For problems that the system cannot solve,
which are often the problems in the next lesson, it will not be possible to
produce a parse
tree.  In this situation VanLehn applies top-down and bottom-up parsing to
construct as much of the parse as possible.  However, the resulting parse
with have a gap in it where the child nodes do not meet with parent nodes.
VanLehn then uses an algorithm called ``parse completion'' to fill in the
gaps.

\begin{fancyfigure}
\centerline{\psfig{file=parse.ps,height=3in}}
\caption{An example of parsing a subtraction example
\protect\cite<from>[pp.~125 and 136]{mindbugs}.  Using the rewrite
rules from the top of the figure, the stages in the examples at the
bottom of the figure can be represented as a tree structure.  Note that
the rules are context sensitive---i.e., Sub1Col\mysub{C} $\rightarrow$
Diff\mysub{C} is only applied when the top digit is $\geq$ bottom digit.
The subscript (C) indicates the column in which the operation applies.}
\label{f:parse}
\end{fancyfigure}

For the problems that cannot be solved, there will be a large number of
partial parses. Top-down and bottom-up parsing is used to produce a set of
these ``skeletons''---the parent and child nodes that define the gap in a
tree.  There will be at least one skeleton from each example that could
not be parsed.  Because of all the constraints,
the intersection of all the
skeletons can produce one unique skeleton.  If this is not the case, the
algorithm uses various biases to select a single skeleton. For example, one
bias is to select the skeleton with the lowest parent nodes.


In developing the constrains on the learner, VanLehn stresses the
importance of a well formed lesson sequence.  Convention allows the learner
to assume that lessons will obey certain rules \cite{vanllear}.  The lesson
should introduce just one ``knowledge chunk'' (disjunction or
subprocedure), and this will often allow the learner to solve problems that
it could not solve before the lesson.  VanLehn suggests that ``\ldots
experienced teachers generate such lesson sequences naturally, without even
realizing that their lesson sequences obey the two conventions''
\citeyear[p.~7]{vanllear}.  In short, good teachers give lessons that make
the learner's task easier.

The skeleton defines the goals and subgoals of a rule.  Patterns also need
to be learned to clarify when the rule is to be applied.  For this, VanLehn
uses Mitchell's \citeyear{mitcvers} version spaces
\cite<see also>[chapter~2]{thortech}. Each example
arithmetic problem is defined by a set of position, fact and action
primitives. Patterns may contain factual primitives, such as ``less-than''
or ``Zero?''.  Just three actions are available: writing a character,
erasing a character and overwriting a character with some other character.
See table~\ref{f:procrep} for an example procedure and problem
representation.  These primitives were selected by guess work, and a more
informed approach would require a theory of perception.

Of all the patterns that could be induced, the most specific pattern is the
one that is learned. For example, from the
subtraction shown in figure~\ref{f:abl}b, the following rule will be
learned:
\begin{alltt}
 borrow column = leftmost and left-adjacent
\end{alltt}

\noindent This
is the most specific pattern to describe the borrowing column: the column
that is both leftmost on the page and also adjacent to the current column.
Most specific is defined relative to the granularity of the representation
language, hence ensuring that generalization takes place,
and the learner does not simply
acquire
a list of past problems. There are other technical reasons for why this
constraint is proposed.  When combined with the constraint that patterns
can only contain conjunctive connectives, it can be shown that the version
space is finite, and there exists a fast algorithm for computing the whole
version space \cite[p.~152]{mindbugs}.  From experimentation with the
system, VanLehn discovered that the most specific rules learned are
sufficient for generating impasses.

VanLehn
\citeyear[chapter~6]{mindbugs} describes a number of other restrictions on
pattern learning.
Indeed, many of the assumptions have been left out, and
much of the detail skipped over in this review.  In particular the
justifications
for many of the assumptions have been omitted.  In essence,
though, the learner is a highly constrained inductive mechanism, learning
control structures to parse problems and patterns in terms of a
predefined set of primitives.  The constrains are psychologically
motivated, and strong enough to make the learning problem tractable.


\begin{fancytable}
\small
\begin{alltt}
        (Problem 1)                 {\em Object 1 is a subtraction problem}
        (Column 2) (Column 3)       {\em Objects 2 \& 3 are columns}
        (Part 1 2) (Part 1 3)       {\em The columns are part of the problem}
        (First 1 3)                 {\em Object 3 is the leftmost object}
        (Adjacent 1 2 3)            {\em The columns are adjacent}
        (Cell 4) (Cell 5) (Cell 6)  {\em Objects 4--5 are cells}
        (Digit 4) (Digit 5)         {\em Objects 4 and 5 are digits}
        (Blank 6)                   {\em Object 6 is a blank cell}

        Sub1Col(C) OR
        1. [And (Digit T) (Part-of T C) (First T C)
                (Digit B) (Part-of B C) (Middle B C)
                (Ordered C T B) (Adjacent C T B)
                (Value-of TV T) (Value-of BV B) (LessThan TV BV)
           ->   (Borrow C)

        2. [And (Digit T) (Part-of T C) (First T C)
                (Digit B) (Part-of B C) (Middle B C)
                (Ordered C T B) (Adjacent C TB)
                (Value-of TV T) (Value-of BV B)
                (Less-Than-or-Equal BV TV)
           ->   (Diff C)

        Diff(c) AND
        1. [And (Digit T) (Part-of T C) (First T C)
                (Digit B) (Part-of B C) (Middle B C)
                (Cell A) (Part-of A C) (Last A C)
                (Ordered C T B) (Adjacent C T B) (Ordered C BA)
                (Value-of TV T) (Value-of BV B)
                (AbsoluteDifference TV BV AV)
           ->   (Write AV A)
\end{alltt}
\caption{Part of the problem representation for 57$-$9, and part
of a subtraction procedure, both using Sierra's representation from
\protect\cite[table~3.8 and 3.10]{mindbugs}.}\label{f:procrep}
\end{fancytable}


\subsubsection{The impasse-repair process}

In addition to the learner, the other component of Sierra is the solver.
This part of the model applies the learned rules to specific problems.
Young \& O'Shea built their rulebases in such a way as to ensure that
the condition patterns would be appropriate for all situations after
conflict resolution. Building on the work of \citeA{repair}, Sierra
incorporates the notion of an {\em impasse}.  For example, when no single
rule is uniquely specified, and impasse is said to have occurred, and
Sierra {\em repairs} the impasse with a number of local modifications to
the solver's state.  That particular case, where more than one rule
matches, is
an instance of a
decision impasse, and appears to be just another kind of conflict
resolution.  In fact VanLehn suggests that there are
three kinds of impasse:
\begin{enumerate}
\item Decision, when a decision is needed, but cannot be made.
\item Reference, for an object in a pattern that is not
uniquely specified.
\item Primitive, where some primitive operation cannot be carried
out---e.g., a student trying to solve 0-1, but not knowing the answer.
\end{enumerate}
These three kinds of impasse are all that Sierra needs to model children's
errors in subtraction.

When an impasse occurs, the local problem solver has to make a repair to
the current state so that Sierra can continue with the problem.  VanLehn
identifies three kinds of repairs:
\begin{enumerate}
\item No-op, which simply skips the offending operation.
\item Barge-on, by relaxing the conditions and then applying the rule.
\item Back-up, by going back and changing a previous action.
\end{enumerate}

Again, just these three kinds of repair are needed to model subtraction and
other problems \cite[p.~43]{mindbugs}. The ``Cartesian product'' of repairs
and impasses (each repair applied to each impasse) should be the repair
strategies observed in the bug data. This assumes that repair strategies
are not limited to specific impasses, but are general methods that could be
applied to many different impasses. \citeA[pp.~44--54]{mindbugs} concludes
that although there are some biases favouring particular repairs for
particular bugs, repair selection can be approximated by random choice.
This suggests a criterion for deciding whether or not to
include new kinds of impasses or
repairs: the new impasse or repair, when multiplied in with the other
impasses and repairs should predict plausible bugs and no implausible bugs.
 Indeed, VanLehn reports (p.~53) that when impasse-repair independence was
first tested, it predicted 16 new bugs, 7 of which have been found.


\begin{fancyfigure}
\centerline{\psfig{file=abl.ps,height=3cm}}
\caption{The bug always-borrows-left (a), where the student
borrows from the leftmost column.  This behaviour is
appropriate for two column problems, such as (b).}\label{f:abl}
\end{fancyfigure}

As an example of the impasse-repair process, consider the subtraction bug
always-borrows-left, shown in figure~\ref{f:abl}a. As noted above, the
skew in the curriculum means that children are exposed to two column
problems before three column problems. Given the learner's bias to learn
the most specific patterns, it acquires the rule ``borrow from the column
that is leftmost and also left-adjacent''.  In the context of two column
problems this is an appropriate rule (see figure~\ref{f:abl}b).  However,
when the child encounters a three column problem which requires borrowing
for the first column, an impasse occurs: there is no column that is both
leftmost and left-adjacent.  At this point a local problem solver takes
control from the interpreter and applies a repair.  If the barge-on repair
is used, one of the rule's conditions is relaxed.  The bug shown in
figure~\ref{f:abl}a results from relaxing the ``left-adjacent''
requirement; relaxing ``leftmost'' clause results
in the correct solution for the subtraction.

\subsubsection{Empirical adequacy}\label{s:emprical}

By the Sierra account, bug
migration is the result of applying different repairs to the same
impasse.  However, the local problem solver
may construct a {\em patch} to
the buggy procedure.  This associates the current repair
with the current impasse, so that when the impasse occurs again, the same
repair will be used.  In this
way both bug migration and stable bugs are accounted for.

The rule sets produced by Sierra (the predictions of the procedures that
children could learn) are tested for bugs.  After each lesson, the results
from the learner are passed to the solver. \citeA[p.~28]{mindbugs} claims
that ``\ldots many bugs are caused by testing beyond training\ldots''.
Hence, the system is tested not only on the current lesson, but also on
harder problems not taught by the current lesson.  Buggy behaviour can also
be found on the current test, especially if the deletion operation has been
applied to a rule set (p.~106). This removes the most
recently added goal from a procedure, resulting in procedures that exhibit
bugs such as \bug{noCoverB}, \bug{stutterA} and \bug{nCA}.
Allowing rules other than the most
recent one to be deleted may result in implausible bugs, e.g., a rule set
without the rules for writing down the answer.


The results from the solver
are diagnosed as buggy or not by passing them to an automated diagnosis
tool called Debuggy \cite{debuggy}.  The overlap between observed bugs and
predicted bugs is the criterion by which Sierra is evaluated. It may be that
Sierra predicts more bugs than have been observed.  This is not a problem
providing that the unobserved bugs are not implausible (star bugs).
VanLehn comments (p.~19):
\begin{ssquote}
When Sierra generates a star bug, it is missing some kind of constraint.
Star bugs indicate that the theory needs revision.  So avoiding the
generation of star bugs is just as important as generating observed bugs.
\end{ssquote}

In the Southbay study of 1147 test solutions, 75 individual bugs were
observed.  Of these, Sierra predicted 28, and failed to predict 47.  Sierra
also predicted 21 plausible bugs which have not yet been observed, and 7
star bugs.  VanLehn argues (p.~200) that when the simulation catches up
with the theory, 39 of the 75 bugs will be predicted (missing 36), with no
star bugs and just 21 plausible (but unobserved) new bugs.
Most of the 36 bugs that are not predicted are pattern errors (e.g.,
N$-$0=0).  If Sierra could generate an impasse when there is a zero to be
borrowed from or to, then these bugs could be explained.  However, VanLehn
notes that ``\ldots empirical quality is not the only measure of
theoretical validity.  \label{p:empirical}
It
must be balanced against explanatory
adequacy\ldots'' \citeyear[p.~205]{mindbugs}.  Young \& O'Shea explain
pattern errors as being derived from confusions from other operations
(i.e., N+0=0).  It is a simple matter to write a production rule with a
condition to capture such pattern errors.  VanLehn is critical of
this method, and demands an explanation of why such errors only occur in
the context of zero and not other numbers.  ``If the model is too easily
tailored, then it is the theorist and not the theory that is doing the
explaining'' \cite[p.~204]{mindbugs}.


\subsection{Summary}


The systematic mistakes made by children solving arithmetic problems
suggests that production systems should be used as a model.
\citeA{younerro} built such a model for subtraction, and demonstrated that
errors could be modelled by small perturbations to the rule set for correct
subtraction.  The modular nature of the system also suggests that learning
could be modelled by adding more rules to the rulebase.  However, the Young
\& O'Shea model did not account for learning.

Repair theory \cite{repair} suggests that buggy core procedures cause
impasses which are repaired from a set of independent repair heuristics.
This principle avoids Young \& O'Shea's arbitrarily hand-coded malrules,
and can also account for bug migration.  In fact repair theory predicted
bug migration before it was found \cite{vanlfeli,vanlbugs}. Sierra theory
\cite{mindbugs} developed from repair theory, and includes a learning
component.  Buggy rules are induced from a skewed curriculum and then
interpreted by a problem solver which detects and repairs impasses.  The
constraints on the theory are argued for with evidence from psychology, and
ensure that the learning task is not only tractable, but also fits the bug
data rather well.

The difference between core procedure bugs and impasses is that the latter
are detectable by the person following the procedure.  However,
\citeA{mindbugs} notes that the existence of impasses is an empirical issue:
ACT* \cite{andearch} doesn't require impasses, but Soar \cite{soar} does.
A discussion of the importance of impasses and the relationship between
Soar and Sierra is deferred until section~\ref{s:important-impasse}.
Despite the subjective nature of labeling errors as bugs or slips, or as
plausible or implausible, it seems that rule based systems are
adequate for modelling children's arithmetic.



\section{Why connectionism?}\label{s:whypdp}


Sierra theory is a success story of symbolic modelling.  Admittedly it is
not without some problems, but it is the most significant existing account
of children's arithmetic and of faulty rule acquisition in general.
Arithmetic looks symbolic and can be modelled with symbolic systems, so why
bother thinking about a connectionist model?  There are at least five
observations which suggest that connectionism is worth considering:

\begin{enumerate}


\item The construction of a connectionist model of arithmetic is a
tough engineering task, but only by implementation do we fully realize the
difficulties faced by connectionism and how they might be solved.

\item There is support for the idea that connectionist systems
are the appropriate tool for capturing developmental phenomena.

\item More generally, connectionist system exhibit
mind-like properties, such as automatic generalization and graceful
degradation.  Connectionist computation is ``brain-style'' computation
\cite{pdp:4}.

\item Theory and implementation are never as independent as one would wish.
Without an alternative for comparison there is a danger that Sierra is
unduly biased by symbolic AI.

\item Connectionism is changing our understanding of notions like
``symbol''.
\end{enumerate}
%
Chapter~\ref{c:fsm} gives details on the construction of the connectionist
model of multicolumn arithmetic. This section expands on the
remaining (strongly interrelated) points.

\subsection{Development}

There is a growing body of research applying connectionism to problems in
developmental psychology.  Examples include general treatments of
developmental phenomena, such as stages of development \cite{shulsimu}, and
models for: balance scale \cite{shulcasc,mcclpara}; seriation
\cite{mareconn}; English verb morphology \cite{plunconn,pdp:18}; and
concept formation and vocabulary growth \cite{plunconn}.  In addition there
are a number of models which look at the importance of development in
terms of constraints which help the learner \cite<most
notably>{elmaincr,elmarepr}. Some authors assert that ``\ldots PDP models
provide a superior account of developmental phenomena than that offered by
cognitivist (symbolic) computational theories'' \cite[p.~1]{plunconn}.
Given this interest in development, it is natural that connectionism should
be applied to arithmetic skills.

It is worth briefly looking at a couple of these models to appreciate
the connectionist approach to development. \citeA{shulsimu} comments
that the vast majority of developmental studies have focussed on what
has developed rather than on how transitions occur.  For example, four
stages have been identified for the balance scale task.  In these
experiments the subject (usually a child) is presented with a beam
balance.  Various weights are placed at various distances from the
fulcrum.  Typically the subject is asked to judge which side of the
balance will go down when a support is removed.  In the first stage
the subject uses weight alone to decide how the scale
balances.  Later, distance from the fulcrum is correctly used, but
only when the weights are equal.  Stage three is characterized by the
correct use of weight and distance in most instances, but confusions
occur when one side has the greater weight and the other side has the
greater distance. Finally, the stage four subject multiplies distance
by weight and compares each side's product to find the answer.

As a starting point to understanding the transitions between the stages,
Shultz identified a set of essential features of stages based on a
previous analysis of Piaget's work.  These features included notions of
qualitative change, stage ordering, and denied any abruptness in transitions.
To elaborate, changes between stages seem to involve a qualitative, and not
quantitative, change.  Transition
is ``\ldots not simply a matter of adding more
information, but rather the emergence of a substantially different way of
processing information'' \cite[p.~105]{shulsimu}.  Stages also tend to be
acquired in a particular order, although stage skipping and regression are
occasionally observed.
The transition from one stage to another is more gradual
than abrupt.  That is, signs of stage $n+1$ performance are present during
stage $n$, and perfection at stage $n+1$ is not achieved until the end of
stage $n+1$.

Shultz listed four ways in which stage development can occur in
connectionist networks.
\begin{enumerate}

\item {\em Hidden unit herding.}  For multilayered networks
(e.g., those trained with backpropagation), each hidden unit does not
select a unique role early in learning. Rather, all the hidden units move
to reduce the current largest error \cite{cascor}.  Eventually hidden
unit responsibilities are sorted out, but in the intervening time stages
can be observed.  An example of this occurs in the learning of past tenses
\cite{pdp:18,plunfrom,marclang}.

\item  {\em Over generalization.}  Hidden unit herding is one form of over
generalization, but a network without hidden units can also
over-generalize.  An initial period of over generalization could be seen as
one
stage, with later stages occurring as the network learned the fine
distinctions in the training set.

\item {\em Training
bias.}  As seen in chapters~\ref{c:xlit} and \ref{c:xnet},
networks are sensitive to problem frequency.
By manipulating the training environment, networks can be made to exhibit
stages.

\item {\em Hidden unit recruitment.} Construction algorithm, such as
cascade correlation \cite{cascor,rcascor}, incrementally
install new hidden units to reduce error.  These kinds of networks are not
pursued further here, but Shultz notes that of 24 hidden units recruited in
his model of the balance scale task, 13 corresponded to stage progressions.
\end{enumerate}

These kinds of connectionist stage transitions also have other properties.
For example, qualitative changes in behaviour are observed, although the
weights of the networks only go through small quantitative changes. It is
noted, though, that whether a change is quantitative or qualitative often
depends on how closely it is looked at.   Likewise, the grain size of a
particular model will determine the degree to which changes are more or
less qualitative.  In connectionist models it is also found that stage
transitions are tentative at first, with the network bobbing between stages
before committing itself. These behaviours are found in both the cascade
correlation \cite{shulcasc} and the backpropagation \cite{mcclpara} models
of the balance scale task, and the model of seriation \cite{mareconn}.
The conclusion drawn is that ``\ldots connectionist modelling of stages is
so far quite consistent with the major regularities in the psychological
literature on cognitive development'' \cite[p.~109]{shulsimu}.


\citeA{elmaincr} has looked at the significance of certain kinds of
development for the learner.  Elman notes that if a child is given just
positive examples of some data, then only a regular grammar can be learned.
 However, natural language appears to belong to a more complex class of
grammars.  Given that children do learn language, the usual assumption is
that there is something innate which constrains the learner to allow it to
learn natural language.  Elman uses the fact that the
learner develops as a constraint which enables it to acquire a
complex grammar.  In one experiment, a network's memory was allowed to grow
over time. Specifically, a simple recurrent network (see
figure~\ref{f:srn}) had its memory ``blanked'' every third or fourth word.
This was done by resetting the context layer every third or fourth input.
The memory span was
increased over time by blanking less often. The network was trained on a
large, complex grammar which included number agreement, use of direct
and optional arguments to verbs, and sentence embedding.  Previously it
had been established that a standard simple recurrent network could not
learn this grammar.  However, by starting with a small memory span and
gradually increasing it, the grammar could be learned.
The network's
memory limitations advantageously constrained what could be learned, in
effect reducing the size of the solution space.
That is ``\ldots they [the memory limitations] act as a filter on the
input, and focus learning on just that subset of facts which lay the
foundations for future success'' \cite[p.~8]{elmaincr}.
This method only applies in structured
environments, in which fragments of the problem are useful in solving the
whole problem. These conditions are not met when learning a random set
of facts.  But Elman comments: ``In practice, the world is not a random
place, and the sorts of things children have to learn about typically
contain a great deal of structure'' \citeyear[p.~9]{elmaincr}.

\begin{fancyfigure}
\centerline{\psfig{file=srn.ps,height=7cm}}
\caption{The simple recurrent network used by \protect\citeA{elmaincr}.
Each word in the lexicon is represented as a 26 bit vector.
The network
learns to re-encode the input into 10 units.  A sentence is fed one word at
a time
to the network.The task is to
predict the next word in the sentence.
At each time step the hidden units are activated and
copied back to the context layer.  Hence, the context layer acts as a
memory of the network's
previous state.  As words are fed in, the hidden layer
uses the current word and the context to build up a representation of the
sentence.}\label{f:srn}
\end{fancyfigure}


Returning to arithmetic, it seems that the symbolic accounts have mainly
been applied to the capture of the end-result of learning.
\citeA[p.~175]{hennwhy} pin-points the problem:
\begin{ssquote}
While the production-system framework has yielded a number of useful
characterisations of children's procedural skills, it has failed so far as
an attempt to model the process involved in development. Its focus is on
describing what is learned---the endstate of a learning process or a
snapshot view---rather than on how malrules are actually generated.
\end{ssquote}
This applies to Sierra, where the learner takes a lesson and a rulebase,
and returns a number of updated rulebases that are consistent with the
lesson.  There is no consideration of how the new rules are incorporated
into the solver, or the effects this has on the system's performance.
VanLehn is developing a theory of impasse-driven learning, but as
discussed in section~\ref{s:important-impasse}, this
is not
yet fully specified. An
inescapable aspect of connectionism is the gradual nature of learning.
There does not have to be ``a moment'' when learning ``happens'', and hence
the snapshot problem of learning in the Young \& O'Shea and the VanLehn
models is side-stepped \cite[p.~14]{bateconn}.


In summary, connectionism looks promising for understanding aspects of
development. As a {\em metaphor}, connectionism does seem better suited
to account for development.  \citeA{bateconn} list the features of symbolic
AI which they believe has hindered the understanding of development:
discrete representations, absolute rules, learning as programming, and the
hardware/software distinction (which places few constraints on what can be
learned).  These points are contrasted with connectionism's distributed
representations, graded rules, learning by structural change, and software
as hardware (the network's knowledge is defined by the structure of the
network). To this list, Bates \& Elman add that the non-linear dynamics
of connectionism can make networks behave in unexpected ways, producing
``truly novel outputs''.  Connectionism does seem to have all the right
properties for capturing development, at least as a metaphor.
Although the metaphor of production systems is
attractive for modelling the stages {\em reached} in development (e.g., as
accumulating rules) connectionist models
provide insight to the {\em transitions} between stages.

The examples above have
demonstrated
that connectionism is useful in understanding development in a number of
domains. It seems reasonable to suppose that a connectionist treatment
could do the same for arithmetic.

\subsection{Implementation and theory}

VanLehn built Sierra as a variant form of a production system.  On making
this decision he comments \citeyear[p.~69]{mindbugs}:
\begin{ssquote}
If one had to choose between production systems and
connection systems as a representation language for knowledge
about subtraction, then production systems seem much more
plausible.  Indeed it is not easy to see how a connection system could
possibly generate the kind of extended, sequential problem-solving
behaviour that characterizes students solving subtraction problems.
\end{ssquote}
It is
hard to disagree with VanLehn here.  Much of this chapter has outlined
the benefits of production system models for arithmetic, and they do indeed
appear to be more plausible.
However, at least one reason why production systems
seem better suited is because there have been few
connectionist models of anything as complicated as arithmetic.
If it were
easy to see how networks can generate ``extended, sequential
problem-solving behaviour'', then perhaps VanLehn would not opt for
production systems so quickly.  This seems to be borne out
by \citeauthor{andearch}'s comment that
connectionist systems ``\ldots have been applied to such a small range of
tasks that they are totally vague on the issue of control of cognition\ldots
in which the precision of productions systems is most exact''
\citeyear[note~2, p.~307]{andearch}.

This section looks at the consequences of selecting a particular
representation language, and concludes that a {\em theory} can be unduly
restricted by the language.  If the goal is to model a phenomenon with a
virtual machine, then although symbolic AI has proved to be useful, there
is no decisive reason to cling to our current understanding of symbolic
representation when designing the virtual machine. Section~\ref{s:symbol}
demonstrates that there are other (connectionist) structures which can be
considered as ``symbols''.  These new structures and new styles of
processing provide a different way to think about problems, and it is
suggested that models of arithmetic will benefit from such ideas.

In selecting production systems, one is typically forced to
build models that rely on the methods of symbolic AI.\@ Rules, frames,
plans, searching, and so on, have
a role to play, but ``the space of computational possibilities has hardly
been entered'' \cite[p.~260]{bodecomp}.  This leaves the following
open question: what kinds of computations are going to be useful in
modelling the mind?  VanLehn has placed his bet on symbolic AI, and this
has no doubt influenced the development of his {\em theory} of arithmetic.
But as \citeA[p.~xvi]{zenon} states:
\begin{ssquote}
We cannot accept an operation as basic just because it is
generally available on conventional computers.  The operations built
into production-model computers were chosen for reasons of economics,
whereas the operations available to the mind are to be discovered
empirically.
\end{ssquote}
By modelling with an operation available on
a conventional computer there is a risk of over- or under-estimating that
operation's importance.  In the worst case the model will
incorporate highly improbable mechanisms, such as demanding a processor
which is much faster than the human information processor.
Another way of looking at this is to say that there is a danger of building
``cognitive wheels''.  A cognitive wheel is:
\begin{ssquote}
Any design proposal in cognitive theory\ldots that is profoundly
unbiological, however wizardly and elegant it is as a bit of technology.

\hfill \cite[p.~147]{denncogn}
\end{ssquote}
As \citeA{clararti,clarbiol} notes, although the mind is treated as a black
box system, we at least know that it is a ``naturally occurring back box''.
Hence, a biological metaphor is suggested, in which it is insisted that
cognitive science ``be concerned with the development and testing of only
such computational mechanisms as seem plausible in the light of whatever
biological constraints may be expected to govern emerging natural
structures'' \cite[p.~47]{clarbiol}.  Of course, these constraints are not
detailed.

Logically there is no reason why nature could not have evolved a
conventional computing architecture, in which case symbolic models would
capture human performance exactly \cite{clarconn,micro}.  In fact, some
kind of higher-level ``programming'' language seems essential.  As
\citeA[p.~3]{andearch} notes:
\begin{ssquote}
\ldots it is totally implausible that we have evolved special facilities or
``organs'' for mathematics, chess, computer programming, or sculpture.
People become expert at activities for which there was no possibility of
anticipation in our evolutionary history\ldots
\end{ssquote}
However, given the kludgey way in which evolution constructs solutions
\cite{clarklud}, it seems unlikely that the symbolic architecture would be
{\em just} that one which we use today.


Connectionism cannot claim to be empirically discovering the ``operations
available to the mind'', and is it just as capable of producing cognitive
wheels as symbolic AI\@. However it is clear that connectionism has changed
the way we think about representational issues (discussed further in the
next section). For example, given that connectionism is seen as being most
applicable to the low-level processes, the following quotation suggests
that it will also be essential in understanding the higher-level processes:
\begin{ssquote}
Whatever the basic principles of language representation, they
are not likely to be utterly unrelated to the way or ways that
the nervous system generates visual representations or auditory
representations, or represents spatial maps or motor planning.

\hfill \cite[p.~42]{churneur}
\end{ssquote}
For ``language representation'' read any of your favourite higher-level
functions.  Although these issues are often dismissed as just
implementational
detail, they can have a profound effect on the models begin considered.
If it turns out that some mechanism, X, is a very cheap computation, it
could be used frequently. If X were expensive, then it would probably be
used less often.  So it seems that neither connectionism nor symbolic AI
alone can expect to explain cognition: there is a sense in which all
modelling involves forcing a phenomenon
into the representation language you
use. Some of the phenomena will fit naturally, and other aspects will not.




There seems no easy way to detect ``unlikely mechanisms'', and it is often
only with hindsight that specific mechanisms may be declared as begin
misguided.  One possible example is any mechanism requiring centralized
control.  Connectionist research and neuroscience has suggested that control
is distributed in the brain \cite[pp.~134--135]{pdp:4}.  A good candidate
for a mechanism that is biased by a technology can be found in repair
theory.  In describing the impasse-repair process, \citeA{repair} comment:
``When a constraint or precondition gets violated the student, unlike a
typical computer program, is not apt to just quit'' (p.~381).  What
\citeauthor{repair} have in mind is an analogy to the computer's error
handling mechanism. This is made explicit by \citeA{mindbugs}:
\begin{ssquote}
%When people reach an impasse\ldots they do not simply halt and
%issue and error message, as a computer would (p.~7).
%
When computers reach an impasse, they do not just turn themselves
off\ldots Instead they start executing an error-handling routine instead of
the main program. Similarly when people are executing a procedure and reach
an impasse, they\ldots handle the impasse in some fashion (p.~41).
\end{ssquote}
The idea is that certain errors are detectable in both human and machine,
and once detected something can be done about the error.  As discussed in
section~\ref{s:mmods}, this way of thinking has been successful in modelling
arithmetic.  However, the analogy breaks down for connectionism.  The
notion of an impasse is ill-defined for networks; networks
will continue to produce output whenever
they are given input.  Thanks to properties
such as similarity-based processing and automatic generalization, networks
never ``get stuck'', unable to continue processing.  This throws a whole
new light on the impasse-repair mechanism: perhaps networks can
automatically repair undefined situations just by virtue of having the
right kind of architecture to start with---being a connectionist
network, rather than a production system.  All this depends, of course, on
what networks actually do (chapter~\ref{c:fsm}), and on the psychological
importance of impasses (discussed in section~\ref{s:important-impasse}).

The point to note here is simply that particular technologies suggest
certain kinds of models and theories, and other technologies can force
different interpretations. Despite his attempt to isolate his theory from
his model, VanLehn concedes that at least one part of his argument ``\ldots
relies on concepts and distinctions from traditional serial computer
science\ldots'' \cite[p.~212]{mindbugs}. He adds: ``\ldots but those
distinctions are rapidly changing as parallel computer science, especially
connectionism, develops''.





\s$5&&&&&&15&&&&&c&3&&23&&49&&&&&&&&&&&&&&&&&&&&& \\
5$\times$2&&&&&2&15&&&&&c&3&&23&&50&&&&&&&&&&&&&&&&&&&&& \\
2$\times$6&&&&&&&105&&&&&c&8&&&25&&&&&&&&&&&&&&&&&&&&& \\
6$\times$2&&&&&&&115&&&&&c&8&&&25&&&&&&&&&&&&2&&&&&&&&& \\
2$\times$7&&&&&&&12&44&&&&48&c&&&&&&&&&&&&&&&3&&&&&&&&& \\
7$\times$2&&&&&&&4&49&&&&49&c&&&&&&&&&&&&&&&4&&&&&&&&& \\
2$\times$8&&5&&&&&&&41&25&&&25&&c&24&&&&&&&&&&&&&&&&&&&&4& \\
8$\times$2&&8&&&&&&&44&25&&&25&&c&25&&&&&&&&&&&&&&&&&&&1&8& \\
2$\times$9&&&&&&&&&&&&10&&&&c&&&&&34&&&&&&&&&&&&&&&& \\
9$\times$2&&&&&&&&&&2&&3&&&&c&&&&&41&&&&&&&&&&&&&&&& \\
3$\times$3&25&&&3&&&&&&c&&&&&&&&&&&50&&&&&&&&&&&&&&&&23\\
3$\times$4&18&&&&&&&&&&&c&&&&25&&&&&32&24&&&&&&&&&&&&&&& \\
4$\times$3&18&&&&&&&&&&&c&&&&22&&&&&38&20&&&&&&&&&&&&&&& \\
3$\times$5&&&&&&&6&&&&23&75&&c&&25&&&25&&&&&&&&&&&&&&&&&& \\
5$\times$3&&&&&&&4&&&&23&77&&c&&25&&&25&&&&&&&&&&&&&&&&&& \\
3$\times$6&&&&&&&29&&&&&85&&&&c&&25&&&&&&&&&&5&&&&&&&&& \\
6$\times$3&&&&&&&27&&&&&75&&&&c&&25&5&&&&&&&&&3&&&&&&&&& \\
3$\times$7&17&&&&&&&&&&&60&&&&20&&c&16&&&47&&&&&&2&&&&&&7&&& \\
7$\times$3&15&&&&&&&&&&&61&&&&15&&c&18&&&47&&&&&&2&&&&&&12&&& \\
3$\times$8&&&&&&&&&&&&&&&&&&&c&&76&&&&&&&&&&&&&&&10&16\\
8$\times$3&&&&&&&&&&&&&&&&&&&c&&72&&&&&&&&&&&&&&&5&13\\
3$\times$9&&&&&&&&&&3&&&&&&14&&&6&&c&&&&&&&&&&&&&&&&5\\
9$\times$3&&&&&&&&&&2&&&&&&15&&&8&&c&&&&&&&&&&&&&&&&5\\
4$\times$4&&11&&&&&&&&29&&25&&7&c&&&&&38&&&&&&35&&&&&&&&&&& \\
4$\times$5&11&&&&&&&&&&&&&25&&&c&&&&&&24&&&&&&19&&&&&&&& \\
5$\times$4&2&&&&&&&&&&&&&25&&&c&&&&&&25&&&&&&26&&&&&&&& \\
4$\times$6&11&&&&&&&&&&&10&&&&&25&&c&&&&9&&&&&4&&&&&&&&& \\
6$\times$4&7&&&&&&&&&&&16&&&&&25&&c&&&&34&&&&&8&&&&&&&&& \\
4$\times$7&29&&&&&&&&&&&&&&&&&29&25&&&c&&&&&&10&&&25&&&&&& \\
7$\times$4&30&&&&&&&&&&&&&&&&&30&25&&&c&&&&&&14&&&25&&&&&& \\
4$\times$8&&&&&&&&&&&&&&&34&&&&2&&&&&c&&&21&&25&&&&&&&& \\
8$\times$4&&&&&&&&&&&&&&&33&&&&5&&&&&c&&&17&&25&&&&&&&2& \\
4$\times$9&&&&&&&&&&&&&&&&&&&&&40&&&&&c&&&25&&&3&&&&& \\
9$\times$4&&&&&&&&&&&&&&&&&&&&&44&&&&&c&&&25&&&2&&&&& \\
5$\times$5&7&&&&&&&&&&&&&&&&&&&c&&&&&&2&&&38&&&&&&&& \\
5$\times$6&&&&&&&&&&&&&&&&&&&&&&&c&&&&&&&&&9&&&&& \\
6$\times$5&&&&&&&&&&&&&&&&&&&&&&&c&&&&&&&&&14&&&&& \\
5$\times$7&25&&&&&&&&&&&&&&&&&&&&&&25&&c&&&25&&&&&&&&& \\
7$\times$5&25&&&&&&&&&&&&&&&&&&&&&&25&&c&&&25&&&&&&&&& \\
5$\times$8&&&&&&&&&&&&&&&&&&&&&&&&&&&c&&80&&&&&&&& \\
8$\times$5&&&&&&&&&&&&&&&&&&&&&&&&&&&c&&68&&&&&&&& \\
5$\times$9&&&&&&&&&&&&&&&&&&&&&&&&&&&&&c&&&25&&&&& \\
9$\times$5&&&&&&&&&&&&&&&&&&&&&&&&&&&&&c&&&23&&&&& \\
6$\times$6&23&&&&&&&&&19&&&&&&&&&&&5&&&&&c&&&17&9&&52&&&&& \\
6$\times$7&&&&&&&&&&&&&&&&&&&&&&&&&&&&c&&23&&18&&&&& \\
7$\times$6&&&&&&&&&&&&&&&&&&&&&&&&&&&&c&&21&&16&&&&& \\
6$\times$8&&&&&&&&&&&&&&&&&&&&&&&&&&&&&&c&&87&&&&& \\
8$\times$6&&&&&&&&&&&&&&&&&&&&&&&&&&&&8&&c&&88&&&&& \\
6$\times$9&&&&&&&&&&&&&&&&&&&&&&&&&&&1&&&&&c&&&&& \\
9$\times$6&&&&&&&&&&&&&&&&&&&&&&&&&&&1&&&&&c&&&&& \\
7$\times$7&46&&&&&&&&&&&&&&&&&&&&&&&&&25&&24&&25&c&5&15&7&&& \\
7$\times$8&25&&&&&&&&&&&&&&&&&&&&&&&&&&&&&27&10&&c&90&&& \\
8$\times$7&25&&&&&&&&&&&&&&&&&&&&&&&&&&&&&34&14&&c&89&&& \\
7$\times$9&14&&&&&&&&&&&&&&&&&&&&&&&&&&10&&&9&&16&&c&&& \\
9$\times$7&13&&&&&&&&&&&&&&&&&&&&&&&&&&14&&&10&&13&&c&&& \\
8$\times$8&&9&&&&&&&23&&&&&&&&&&&&&&&&&&&&&9&10&&&&c&50&31\\
8$\times$9&&&&&&&&&&&&&&&&&&&&&&&&&&&&&&&&&&23&&c& \\
9$\times$8&&&&&&&&&&&&&&&&&&&&&&&&&&&&&&&&&&24&&c& \\
9$\times$9&&&&&&&&&&&&&&&&&&&&&&&&&&&&&9&&&&&&7&19&c\\
\end{tabular}
\caption{Errors of 20 ``skewed'' networks.  No errors were made on
zero problems, and these have been omitted from the table. ``c'' means
correct answer, and on average represents 423 correct recalls from 500
trials.}
\label{f:netematzn}
\end{fancytable}
%\end{table}
richardd/papers/dphil/conc.aux   644    517    517        2072  5440156315  16417 0ustar  richarddrichardd\relax 
\citation{mcclmath}
\@writefile{toc}{\string\contentsline\space {chapter}{Chapter \string\numberline\space {6}Summary}{162}}
\@writefile{lof}{\string\addvspace\space {10\p@ }}
\@writefile{lot}{\string\addvspace\space {10\p@ }}
\@writefile{toc}{\string\contentsline\space {section}{\string\numberline\space {6.1}Memory for arithmetic facts}{162}}
\@writefile{toc}{\string\contentsline\space {section}{\string\numberline\space {6.2}Multicolumn arithmetic}{163}}
\citation{suppproc}
\citation{rosesymb}
\@writefile{toc}{\string\contentsline\space {section}{\string\numberline\space {6.3}Future work}{165}}
\global\@namedef{cp@conc}{
\setcounter{page}{166}
\setcounter{equation}{0}
\setcounter{enumi}{2}
\setcounter{enumii}{0}
\setcounter{enumiii}{0}
\setcounter{enumiv}{0}
\setcounter{footnote}{0}
\setcounter{mpfootnote}{0}
\setcounter{part}{2}
\setcounter{chapter}{6}
\setcounter{section}{3}
\setcounter{subsection}{0}
\setcounter{subsubsection}{0}
\setcounter{paragraph}{0}
\setcounter{subparagraph}{0}
\setcounter{figure}{0}
\setcounter{table}{0}
\setcounter{APAenum}{0}
}
richardd/papers/dphil/miscount.p  