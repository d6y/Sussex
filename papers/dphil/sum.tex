\documentstyle{richardd}

\def\x#1#2{\mbox{#1$\times$#2}}

\begin{document}

\title{Dynamics of Arithmetic\\A Connectionist View of Arithmetic Skills}
\author{Richard Dallaway}
\date{6 July 1993}
\maketitle
\thispagestyle{empty}

\noindent Arithmetic
takes time.  Children need five or six years to master the one
hundred multiplication facts (\x00 to \x99), and it takes adults
approximately one second to recall an answer to a problem like \x78.
Multicolumn arithmetic (e.g., \x{45}{67}) requires a sequence of actions,
and children produce a host of systematic mistakes when solving such
problems. This thesis models the time course and mistakes of people solving
arithmetic problems.  Two models are presented, both of which are built
from connectionist components.

First, a model of memory for multiplication facts is described. Here a
system is built to capture the response time and slips of adults recalling
two digit multiplication facts. The phenomenon is thought of as spreading
activation between problem nodes (such as ``7'' and ``8'') and product
nodes (``56''). Results are presented from simulations designed to mimic
experiments on adults' memory for single-digit multiplication. The details
and assumptions of the model (response time measurement, learning
assumptions) are examined, and the simulation results are compared to the
results found in the psychological literature. It is found that a simple
model can account for the phenomena, and the implications are discussed in
light of a number of similar models.

Second, an account is given of children's errors in multicolumn
multiplication. To date, only production system models have been used in
this domain. The aim is not to produce a detailed fit to the
empirical observations of errors, but to demonstrate how a connectionist
system can model the behaviour, and what advantages this brings. The style
of the model moves away from the received wisdom of an impasse-repair
process (when a child encounters a problem an impasse is said to have
occurred, which is then repaired with general-purpose heuristics).  Rather,
the view presented is of a system in which processing is similarity-based,
invoking a ``processing trajectory'' style of explanation. That is,
instead of strictly following rules to solve arithmetic problems, the
system traverses states; rather than acquiring rules, the system's state
space evolves.  The result is a {\em graded state machine}, a system with
some of the properties of finite state machines, but with the additional
flexibility of connectionist networks.  The analysis of the system shows
that connectionist representations can be structured in ways that are
useful for modelling procedural skills (such as arithmetic). It is
suggested that
one of the strengths of the model is its emphasis on development, rather
than on ``snap-shot'' accounts. The system is compared to symbolic models,
and the notions like ``impasse'' and ``repair'' are discussed
from a connectionist perspective.

\end{document}
