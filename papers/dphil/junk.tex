
Cox tested children on the four multicolumn tasks by having them complete
test which were based around a number of levels.  For addition there were
eight levels, starting from addition of two digits to one digit without
renaming.  Each level became increasingly difficult, up to addition of
three two-digit numbers with renaming.  Cox listed the bugs found at each
level, and then over the levels produced a categorization of the bugs.  As
can be seen from figure~\ref{f:coxcat}, Cox classified the bugs according
to the kind of faulty knowledge that caused the error.
The meaning behind Cox's labels is obvious, except for ``concept'', which
is the case when the child seem to be lacking a basic understanding of the
concept of multiplication or addition.

Discussion of percentages goes here


\begin{fancytable}
\begin{tabular}{lr@{\hspace{0.75in}}lr}
\multicolumn{2}{c}{Addition}&\multicolumn{2}{c}{Multiplication}\\
Renaming        & 23 &  Concept         & 19\\
Concept         & 17 &  Partial product & 13\\
Wrong operation & 6  & $\times$ after renaming & 10\\
Place value     & 5  & + after renaming & 7\\
& & Renaming        & 6\\
& & $\times$ by zero & 6\\
& & Wrong operation & 4\\
& & Reversal of digits & 2\\
\end{tabular}
\caption{Categories of bugs reported by \protect\citeA{coxanal},
and the number of bugs that
fell under the category heading.}
\label{f:coxcat}
\end{fancytable}

---------------------------------------------------------------------


I have no doubt that humans do production-system-like things
\cite{hadlconn}, but which things? and to what degree? For example, in
production systems variable binding is used all the time (in fact,
production systems do little more than bind values to variables).  On the
question of how important variable binding is to production system models,
there is nothing to be gained by saying that it is ``just implementational
detail''. Variable binding is a essential element of production systems.
If you remove variable binding from production systems, you'll have to
specify something to go in its place. Judging from the problems
connectionists have had with variable binding
\cite{shassimp,starsymb,starlear},
it's not
obvious what that something should be.

\citeA{kirspric} discusses the idea that ``\ldots variables are costly to
implement in PDP systems and hence are not likely to be as important in
cognitive processing as orthodox theories of cognition assume'' (p.~119).
Kirsh concludes that there is no strong argument here. ``Nonetheless, I
think connectionism justly forces us to reconsider whether there are
alternative formulations of the problems the cognitive system solves,
formulations that lend themselves to solutions without much variable value
paring either explicitly or implicitly'' (p.~133).
