\par\bigskip\par\bugitem{ZXnEZCn}{0$\times$N=0-carry-N} \nopagebreak When multiplying by zero, zero is written as the column's answer, but
 the multiplicand is carried.\nopagebreak\par\nopagebreak\medskip\nopagebreak 
\begin{arithprob}{p{1em}p{1em}p{1em}}
$\ _{\ }$&$2_{\ }$&$0_{\ }$\\
$\times$$\ _{\ }$&$\ _{\ }$&$3_{\ }$\\
\cline{1-3}$\ _{\ }$&$9_{3}$&$0_{\ }$\\
\end{arithprob}
\hfil\begin{tabular}[t]{lr}Ainsworth&2.63\%\\\end{tabular}\par\bigskip\bugitem{ACtotop}{adds-carry-and-multiplicand} \nopagebreak The carried digit is added to the multiplicand, and this sum is given
 as the column answer.  E.g., \x68=48, 3+4=7.  The final ``5'' was
 copied.\nopagebreak\par\nopagebreak\medskip\nopagebreak 
\begin{arithprob}{p{1em}p{1em}p{1em}p{1em}}
$\ _{\ }$&$5_{\ }$&$3_{\ }$&$6_{\ }$\\
$\times$$\ _{\ }$&$\ _{\ }$&$\ _{\ }$&$8_{\ }$\\
\cline{1-4}$\ _{\ }$&$5_{\ }$&$7_{4}$&$8_{\ }$\\
\end{arithprob}
\hfil\begin{tabular}[t]{lr}Attisha&\\\end{tabular}\par\bigskip\bugitem{ACtobot}{adds-carry-and-multiplier} \nopagebreak The carried digit is added to the multiplier, and this sum is given
 as the column answer.  I.e., \x45=20, 4+2=6, \x48=32. \nopagebreak\par\nopagebreak\medskip\nopagebreak 
\begin{arithprob}{p{1em}p{1em}p{1em}p{1em}}
$\ _{\ }$&$8_{\ }$&$0_{\ }$&$5_{\ }$\\
$\times$$\ _{\ }$&$\ _{\ }$&$\ _{\ }$&$4_{\ }$\\
\cline{1-4}$3_{\ }$&$2_{\ }$&$6_{2}$&$0_{\ }$\\
\end{arithprob}
\hfil\begin{tabular}[t]{lr}Cox&8.85\%\\\end{tabular}\par\bigskip\bugitem{ACtobotZ}{adds-carry-and-multiplier-when-zero} \nopagebreak When the multiplicand is a zero, the subject adds the carry digit
 and the multiplier to obtain an answer.  In the example, \x27=14,
 1+2=3, \x25=10.\nopagebreak\par\nopagebreak\medskip\nopagebreak 
\begin{arithprob}{p{1em}p{1em}p{1em}p{1em}}
$\ _{\ }$&$5_{\ }$&$0_{\ }$&$7_{\ }$\\
$\times$$\ _{\ }$&$\ _{\ }$&$\ _{\ }$&$2_{\ }$\\
\cline{1-4}$1_{\ }$&$0_{\ }$&$3_{1}$&$4_{\ }$\\
\end{arithprob}
\hfil\begin{tabular}[t]{lr}Cox&0.88\%\\\end{tabular}\par\bigskip\bugitem{ACtomultiplicands}{adds-carry-to-multiplicands} \nopagebreak A column's answer is the sum of the carry digit and the multiplicand.
 E.g., \x68=48, 3+4=7, 5+4=9.\nopagebreak\par\nopagebreak\medskip\nopagebreak 
\begin{arithprob}{p{1em}p{1em}p{1em}p{1em}}
$\ _{\ }$&$5_{\ }$&$3_{\ }$&$6_{\ }$\\
$\times$$\ _{\ }$&$\ _{\ }$&$\ _{\ }$&$8_{\ }$\\
\cline{1-4}$\ _{\ }$&$9_{\ }$&$7_{4}$&$8_{\ }$\\
\end{arithprob}
\hfil\begin{tabular}[t]{lr}Attisha&\\\end{tabular}\par\bigskip\bugitem{ACtoP}{adds-carry-to-product} \nopagebreak When the result of a multiplication is a two digit number, those
 numbers are added, e.g., \x35=15=6.\nopagebreak\par\nopagebreak\medskip\nopagebreak 
\begin{arithprob}{p{1em}p{1em}p{1em}p{1em}}
$\ _{\ }$&$\ _{\ }$&$5_{\ }$&$2_{\ }$\\
$\ _{\ }$&$\times$$\ _{\ }$&$1_{\ }$&$3_{\ }$\\
\cline{2-4}$\ _{\ }$&$\ _{\ }$&$6_{\ }$&$6_{\ }$\\
$+$$\ _{\ }$&$5_{\ }$&$2_{\ }$&$0_{\ }$\\
\cline{1-4}$\ _{\ }$&$5_{\ }$&$8_{\ }$&$6_{\ }$\\
\end{arithprob}
\hfil\begin{tabular}[t]{lr}Ainsworth&1.32\%\\\end{tabular}\par\bigskip\bugitem{AinsteadX}{adds-instead-of-multiplying} \nopagebreak The addition algorithm is used instead of multiplication.\nopagebreak\par\nopagebreak\medskip\nopagebreak 
\begin{arithprob}{p{1em}p{1em}p{1em}p{1em}}
$\ _{\ }$&$7_{\ }$&$2_{\ }$&$5_{\ }$\\
$\times$$\ _{\ }$&$\ _{\ }$&$\ _{\ }$&$3_{\ }$\\
\cline{1-4}$\ _{\ }$&$7_{\ }$&$2_{\ }$&$8_{\ }$\\
\end{arithprob}
\hfil\begin{tabular}[t]{lr}Cox&2.65\%\\Attisha&\\\end{tabular}\par\bigskip\bugitem{multAtoN}{adds-multiplicand-to-answer} \nopagebreak A multiplicand is not multiplied, but instead is added to the answer. 
I.e., \x36=18, 7+1=8.\nopagebreak\par\nopagebreak\medskip\nopagebreak 
\begin{arithprob}{p{1em}p{1em}p{1em}}
$\ _{\ }$&$7_{\ }$&$6_{\ }$\\
$\times$$\ _{\ }$&$\ _{\ }$&$3_{\ }$\\
\cline{1-3}$\ _{\ }$&$8_{1}$&$8_{\ }$\\
\end{arithprob}
\hfil\begin{tabular}[t]{lr}Buswell&5.47\%\\\end{tabular}\par\bigskip\bugitem{AinXpattern}{adds-using-multiplication-pattern} \nopagebreak The subject uses the pattern for multiplication, but adds the digits.\nopagebreak\par\nopagebreak\medskip\nopagebreak 
\begin{arithprob}{p{1em}p{1em}p{1em}p{1em}}
$\ _{\ }$&$3_{\ }$&$2_{\ }$&$0_{\ }$\\
$\times$$\ _{\ }$&$\ _{\ }$&$\ _{\ }$&$4_{\ }$\\
\cline{1-4}$\ _{\ }$&$7_{\ }$&$6_{\ }$&$4_{\ }$\\
\end{arithprob}
\hfil\begin{tabular}[t]{lr}Cox&1.77\%\\Attisha&\\\end{tabular}\par\bigskip\bugitem{alwaysC}{always-carries} \nopagebreak The subject always adds in the carry digit.\nopagebreak\par\nopagebreak\medskip\nopagebreak 
\begin{arithprob}{p{1em}p{1em}p{1em}p{1em}}
$2_{\ }$&$4_{\ }$&$2_{\ }$&$9_{\ }$\\
$\ _{\ }$&$\ _{\ }$&$\ _{\ }$&$2_{\ }$\\
\cline{1-4}$5_{\ }$&$9_{\ }$&$5_{1}$&$8_{\ }$\\
\end{arithprob}
\hfil\begin{tabular}[t]{lr}Buswell&0.2\%\\Attisha&\\\end{tabular}\par\bigskip\bugitem{alwaysCone}{always-carries-one} \nopagebreak When a carry occurs, the subject adds one to a column answer, not the
 real carry.\nopagebreak\par\nopagebreak\medskip\nopagebreak 
\begin{arithprob}{p{1em}p{1em}p{1em}p{1em}}
$\ _{\ }$&$5_{\ }$&$1_{\ }$&$4_{\ }$\\
$\times$$\ _{\ }$&$\ _{\ }$&$\ _{\ }$&$7_{\ }$\\
\cline{1-4}$3_{\ }$&$5_{\ }$&$8_{2}$&$8_{\ }$\\
\end{arithprob}
\hfil\begin{tabular}[t]{lr}Attisha&\\\end{tabular}\par\bigskip\bugitem{Nonerow}{answer-on-one-row} \nopagebreak All the partial products are written on one answer row.\nopagebreak\par\nopagebreak\medskip\nopagebreak 
\begin{arithprob}{p{1em}p{1em}p{1em}p{1em}}
$\ _{\ }$&$\ _{\ }$&$2_{\ }$&$3_{\ }$\\
$\ _{\ }$&$\times$$\ _{\ }$&$4_{\ }$&$8_{\ }$\\
\cline{2-4}$9_{1}$&$3_{1}$&$8_{2}$&$4_{\ }$\\
\end{arithprob}
\hfil\begin{tabular}[t]{lr}Ainsworth&27.63\%\\\end{tabular}\par\bigskip\bugitem{lrN}{answers-left-to-right} \nopagebreak The subject writes the answer left to right.  In the example, \x29=18,
 subject writes 8 carries 1, and so on.\nopagebreak\par\nopagebreak\medskip\nopagebreak 
\begin{arithprob}{p{1em}p{1em}p{1em}p{1em}}
$\ _{\ }$&$7_{\ }$&$1_{\ }$&$2_{\ }$\\
$\times$$\ _{\ }$&$\ _{\ }$&$\ _{\ }$&$9_{\ }$\\
\cline{1-4}$8_{\ }$&$0_{1}$&$6_{1}$&$4_{\ }$\\
\end{arithprob}
\hfil\begin{tabular}[t]{lr}Attisha&\\\end{tabular}\par\bigskip\bugitem{CwrongX}{carries-wrong-digit} \nopagebreak When the result of a multiplication or addition is a number that needs
 to be carried, the wrong digit is carried.\nopagebreak\par\nopagebreak\medskip\nopagebreak 
\begin{arithprob}{p{1em}p{1em}p{1em}p{1em}}
$\ _{\ }$&$7_{\ }$&$2_{\ }$&$4_{\ }$\\
$\times$$\ _{\ }$&$\ _{\ }$&$\ _{\ }$&$6_{\ }$\\
\cline{1-4}$4_{\ }$&$8_{6}$&$1_{4}$&$2_{\ }$\\
\end{arithprob}
\hfil\begin{tabular}[t]{lr}Ainsworth&3.95\%\\Buswell&1.76\%\\Cox&0.88\%\\Attisha&\\\end{tabular}\par\bigskip\bugitem{CwrongNum}{carries-wrong-number} \nopagebreak A composite bug, where some number was carried, but it was the wrong
 one (e.g., the units number as in \bug{CwrongX}, or always a one, as in
 \bug{alwaysCone}).\nopagebreak\par\nopagebreak\medskip\nopagebreak 
\begin{tabular}[t]{lr}Buswell&18.55\%\\\end{tabular}\par\bigskip\bugitem{CAbeforeX}{carry-added-to-multiplicand} \nopagebreak The carry digit is added to the multiplicand before multiplying. I.e.,
 \x67=42, (2+4)$\times$6=36, (3+3)$\times$6=36. \nopagebreak\par\nopagebreak\medskip\nopagebreak 
\begin{arithprob}{p{1em}p{1em}p{1em}p{1em}}
$\ _{\ }$&$3_{\ }$&$2_{\ }$&$7_{\ }$\\
$\times$$\ _{\ }$&$\ _{\ }$&$\ _{\ }$&$6_{\ }$\\
\cline{1-4}$3_{\ }$&$6_{3}$&$6_{4}$&$2_{\ }$\\
\end{arithprob}
\hfil\begin{tabular}[t]{lr}Cox&7.96\%\\Buswell&0.78\%\\Attisha&\\\end{tabular}\par\bigskip\bugitem{CAtotens}{carry-added-to-tens} \nopagebreak When adding a carry digit to a product, the carry is added to the
 tens part, e.g., \x46=24, \x42=8, 2+8=28.\nopagebreak\par\nopagebreak\medskip\nopagebreak 
\begin{arithprob}{p{1em}p{1em}p{1em}p{1em}}
$\ _{\ }$&$\ _{\ }$&$2_{\ }$&$6_{\ }$\\
$\ _{\ }$&$\times$$\ _{\ }$&$1_{\ }$&$4_{\ }$\\
\cline{2-4}$\ _{\ }$&$2_{\ }$&$8_{2}$&$4_{\ }$\\
$+$$\ _{\ }$&$2_{\ }$&$6_{\ }$&$0_{\ }$\\
\cline{1-4}$\ _{\ }$&$5_{1}$&$4_{\ }$&$4_{\ }$\\
\end{arithprob}
\hfil\begin{tabular}[t]{lr}Ainsworth&1.32\%\\\end{tabular}\par\bigskip\bugitem{Cnotraised}{carry-not-raised} \nopagebreak The carry digit is not raised at the end of a answer row in 
the partial product.\nopagebreak\par\nopagebreak\medskip\nopagebreak 
\begin{arithprob}{p{1em}p{1em}p{1em}p{1em}}
$\ _{\ }$&$\ _{\ }$&$4_{\ }$&$2_{\ }$\\
$\ _{\ }$&$\times$$\ _{\ }$&$4_{\ }$&$1_{\ }$\\
\cline{2-4}$\ _{\ }$&$\ _{\ }$&$4_{\ }$&$2_{\ }$\\
$+$$\ _{1}$&$6_{\ }$&$8_{\ }$&$0_{\ }$\\
\cline{1-4}$\ _{\ }$&$7_{1}$&$2_{\ }$&$2_{\ }$\\
\end{arithprob}
\hfil\begin{tabular}[t]{lr}Ainsworth&1.32\%\\\end{tabular}\par\bigskip\bugitem{ConeCallX}{carry-once-always-carry} \nopagebreak Once the subject starts to carry a digit, it is always carried.\nopagebreak\par\nopagebreak\medskip\nopagebreak 
\begin{arithprob}{p{1em}p{1em}p{1em}p{1em}}
$\ _{\ }$&$1_{\ }$&$1_{\ }$&$2_{\ }$\\
$\times$$\ _{\ }$&$\ _{\ }$&$\ _{\ }$&$7_{\ }$\\
\cline{1-4}$\ _{\ }$&$8_{1}$&$8_{1}$&$4_{\ }$\\
\end{arithprob}
\hfil\begin{tabular}[t]{lr}Cox&0.88\%\\\end{tabular}\par\bigskip\bugitem{copyafterfirst}{copies-after-first-column} \nopagebreak The first column of a problem is solved correctly, but the
 remaining multiplicands are copied to the answer row.\nopagebreak\par\nopagebreak\medskip\nopagebreak 
\begin{arithprob}{p{1em}p{1em}p{1em}p{1em}}
$\ _{\ }$&$3_{\ }$&$1_{\ }$&$3_{\ }$\\
$\times$$\ _{\ }$&$\ _{\ }$&$\ _{\ }$&$3_{\ }$\\
\cline{1-4}$\ _{\ }$&$3_{\ }$&$1_{\ }$&$9_{\ }$\\
\end{arithprob}
\hfil\begin{tabular}[t]{lr}Cox&7.96\%\\Attisha&\\\end{tabular}\par\bigskip\bugitem{copymultiplicand}{copies-multiplicand} \nopagebreak No multiplication is performed, but the multiplicand is copied to the
 answer row.\nopagebreak\par\nopagebreak\medskip\nopagebreak 
\begin{arithprob}{p{1em}p{1em}p{1em}p{1em}}
$\ _{\ }$&$2_{\ }$&$0_{\ }$&$0_{\ }$\\
$\times$$\ _{\ }$&$\ _{\ }$&$\ _{\ }$&$4_{\ }$\\
\cline{1-4}$\ _{\ }$&$2_{\ }$&$0_{\ }$&$0_{\ }$\\
\end{arithprob}
\hfil\begin{tabular}[t]{lr}Cox&4.42\%\\Buswell&0.78\%\\Attisha&\\\end{tabular}\par\bigskip\bugitem{copy100}{copies-multiplicand-at-100s} \nopagebreak When processing the hundreds multiplier, the subject inserts two 
 zeros and copies the multiplicand.\nopagebreak\par\nopagebreak\medskip\nopagebreak 
\begin{arithprob}{p{1em}p{1em}p{1em}p{1em}p{1em}}
$\ _{\ }$&$\ _{\ }$&$5_{\ }$&$1_{\ }$&$9_{\ }$\\
$\ _{\ }$&$\times$$\ _{\ }$&$4_{\ }$&$0_{\ }$&$2_{\ }$\\
\cline{2-5}$\ _{\ }$&$1_{\ }$&$0_{\ }$&$3_{\ }$&$8_{\ }$\\
$5_{\ }$&$1_{\ }$&$9_{\ }$&$0_{\ }$&$0_{\ }$\\
\end{arithprob}
\hfil\begin{tabular}[t]{lr}Cox&0.88\%\\\end{tabular}\par\bigskip\bugitem{copymultiplicandZ}{copies-multiplicand-including-zero} \nopagebreak The multiplicand is copied as the answer, but a zero is first
 inserted into the answer.\nopagebreak\par\nopagebreak\medskip\nopagebreak 
\begin{arithprob}{p{1em}p{1em}p{1em}p{1em}}
$\ _{\ }$&$2_{\ }$&$4_{\ }$&$7_{\ }$\\
$\times$$\ _{\ }$&$\ _{\ }$&$2_{\ }$&$0_{\ }$\\
\cline{1-4}$2_{\ }$&$4_{\ }$&$7_{\ }$&$0_{\ }$\\
\end{arithprob}
\hfil\begin{tabular}[t]{lr}Cox&0.88\%\\\end{tabular}\par\bigskip\bugitem{copymultiplicandless2}{copies-multiplicand-less-2} \nopagebreak The answer is two less than the multiplicand.\nopagebreak\par\nopagebreak\medskip\nopagebreak 
\begin{arithprob}{p{1em}p{1em}p{1em}}
$\ _{\ }$&$1_{\ }$&$6_{\ }$\\
$\times$$\ _{\ }$&$\ _{\ }$&$4_{\ }$\\
\cline{1-3}$\ _{\ }$&$1_{\ }$&$4_{\ }$\\
\end{arithprob}
\hfil\begin{tabular}[t]{lr}Cox&0.88\%\\\end{tabular}\par\bigskip\bugitem{crossX}{cross-multiplies} \nopagebreak The digits of the problem are cross multiplied, e.g., \x14=4, \x32=6.\nopagebreak\par\nopagebreak\medskip\nopagebreak 
\begin{arithprob}{p{1em}p{1em}p{1em}}
$\ _{\ }$&$4_{\ }$&$2_{\ }$\\
$\times$$\ _{\ }$&$3_{\ }$&$1_{\ }$\\
\cline{1-3}$\ _{\ }$&$6_{\ }$&$4_{\ }$\\
\end{arithprob}
\hfil\begin{tabular}[t]{lr}Ainsworth&1.32\%\\\end{tabular}\par\bigskip\bugitem{digitomit}{digit-omitted} \nopagebreak A digit in the product is not written down.  In the example, the subject
 decided not to write down the 5 from 54 (\x88=64, \x86=48 + 6 = 54).\nopagebreak\par\nopagebreak\medskip\nopagebreak 
\begin{arithprob}{p{1em}p{1em}p{1em}p{1em}p{1em}}
$\ _{\ }$&$\ _{\ }$&$\ _{\ }$&$6_{\ }$&$8_{\ }$\\
$\times$$\ _{\ }$&$9_{\ }$&$8_{\ }$&$7_{\ }$&$8_{\ }$\\
\cline{1-5}$\ _{\ }$&$\ _{\ }$&$\ _{\ }$&$4_{6}$&$4_{\ }$\\
\end{arithprob}
\hfil\begin{tabular}[t]{lr}Buswell&3.32\%\\\end{tabular}\par\bigskip\bugitem{dnAC}{does-not-add-carry} \nopagebreak The carry digit is not added to the column product.
 Cox notes this error when the subject misses just one carry in a problem
 (not necessarily every carry).\nopagebreak\par\nopagebreak\medskip\nopagebreak 
\begin{arithprob}{p{1em}p{1em}p{1em}p{1em}}
$\ _{\ }$&$1_{\ }$&$4_{\ }$&$9_{\ }$\\
$\times$$\ _{\ }$&$\ _{\ }$&$\ _{\ }$&$4_{\ }$\\
\cline{1-4}$\ _{\ }$&$4_{1}$&$6_{3}$&$6_{\ }$\\
\end{arithprob}
\hfil\begin{tabular}[t]{lr}Buswell&17.38\%\\Cox&4.42\%\\Attisha&\\\end{tabular}\par\bigskip\bugitem{dnApP}{does-not-add-partial-product} \nopagebreak The subject does not add the partial product, leaving the sum as shown
 in the example.\nopagebreak\par\nopagebreak\medskip\nopagebreak 
\begin{arithprob}{p{1em}p{1em}p{1em}p{1em}p{1em}}
$\ _{\ }$&$\ _{\ }$&$\ _{\ }$&$5_{\ }$&$3_{\ }$\\
$\ _{\ }$&$\times$$\ _{\ }$&$3_{\ }$&$2_{\ }$&$1_{\ }$\\
\cline{2-5}$\ _{\ }$&$\ _{\ }$&$\ _{\ }$&$5_{\ }$&$3_{\ }$\\
$\ _{\ }$&$1_{\ }$&$0_{\ }$&$6_{\ }$&$0_{\ }$\\
$1_{\ }$&$5_{\ }$&$9_{\ }$&$0_{\ }$&$0_{\ }$\\
\end{arithprob}
\hfil\begin{tabular}[t]{lr}Buswell&2.34\%\\\end{tabular}\par\bigskip\bugitem{nCpP}{does-not-carry-in-partial-product} \nopagebreak The subject does not carry when adding the partial product.\nopagebreak\par\nopagebreak\medskip\nopagebreak 
\begin{arithprob}{p{1em}p{1em}p{1em}p{1em}p{1em}}
$\ _{\ }$&$\ _{\ }$&$9_{\ }$&$2_{\ }$&$7_{\ }$\\
$\ _{\ }$&$\times$$\ _{\ }$&$\ _{\ }$&$7_{\ }$&$3_{\ }$\\
\cline{2-5}$\ _{\ }$&$2_{\ }$&$7_{\ }$&$8_{\ }$&$1_{\ }$\\
$6_{\ }$&$4_{\ }$&$8_{\ }$&$9_{\ }$&$0_{\ }$\\
\cline{1-5}$6_{\ }$&$6_{\ }$&$5_{\ }$&$7_{\ }$&$1_{\ }$\\
\end{arithprob}
\hfil\begin{tabular}[t]{lr}Attisha&\\\end{tabular}\par\bigskip\bugitem{noCfor10s}{does-not-carry-to-10s} \nopagebreak The carried digit is not added to the product in the tens column.\nopagebreak\par\nopagebreak\medskip\nopagebreak 
\begin{arithprob}{p{1em}p{1em}p{1em}p{1em}}
$\ _{\ }$&$2_{\ }$&$1_{\ }$&$6_{\ }$\\
$\times$$\ _{\ }$&$\ _{\ }$&$\ _{\ }$&$6_{\ }$\\
\cline{1-4}$1_{\ }$&$2_{\ }$&$6_{3}$&$6_{\ }$\\
\end{arithprob}
\hfil\begin{tabular}[t]{lr}Ainsworth&3.95\%\\Cox&0.88\%\\Attisha&\\\end{tabular}\par\bigskip\bugitem{dnRPcopy10s}{does-not-rename-copies-10s} \nopagebreak The product from the first multiplication is written in the answer row
 without renaming, and the tens multiplicand is copied into the answer.\nopagebreak\par\nopagebreak\medskip\nopagebreak 
\begin{arithprob}{p{1em}p{1em}p{1em}}
$\ _{\ }$&$1_{\ }$&$6_{\ }$\\
$\times$$\ _{\ }$&$\ _{\ }$&$4_{\ }$\\
\cline{1-3}$1_{\ }$&$2_{\ }$&$4_{\ }$\\
\end{arithprob}
\hfil\begin{tabular}[t]{lr}Cox&0.88\%\\\end{tabular}\par\bigskip\bugitem{dnRthencopy}{does-not-rename-first-then-copies} \nopagebreak The first multiplication is performed, and the answer is written in the
 answer without renaming, and remaining multiplicands are copied.\nopagebreak\par\nopagebreak\medskip\nopagebreak 
\begin{arithprob}{p{1em}p{1em}p{1em}p{1em}}
$\ _{\ }$&$2_{\ }$&$3_{\ }$&$7_{\ }$\\
$\times$$\ _{\ }$&$\ _{\ }$&$\ _{\ }$&$4_{\ }$\\
\cline{1-4}$2_{\ }$&$3_{\ }$&$2_{\ }$&$8_{\ }$\\
\end{arithprob}
\hfil\begin{tabular}[t]{lr}Attisha&\\\end{tabular}\par\bigskip\bugitem{dnRP}{does-not-rename-product} \nopagebreak Digits carried over from a multiplication are written on the answer row.\nopagebreak\par\nopagebreak\medskip\nopagebreak 
\begin{arithprob}{p{1em}p{1em}p{1em}}
$\ _{\ }$&$1_{\ }$&$7_{\ }$\\
$\times$$\ _{\ }$&$\ _{\ }$&$5_{\ }$\\
\cline{1-3}$5_{\ }$&$3_{\ }$&$5_{\ }$\\
\end{arithprob}
\hfil\begin{tabular}[t]{lr}Ainsworth&6.58\%\\Cox&0.88\%\\\end{tabular}\par\bigskip\bugitem{noleadingZ}{forgets-annex} \nopagebreak The zero is forgotten.  In the example, a zero should
 have been inserted into the second answer row.\nopagebreak\par\nopagebreak\medskip\nopagebreak 
\begin{arithprob}{p{1em}p{1em}p{1em}p{1em}}
$\ _{\ }$&$\ _{\ }$&$4_{\ }$&$5_{\ }$\\
$\ _{\ }$&$\times$$\ _{\ }$&$2_{\ }$&$9_{\ }$\\
\cline{2-4}$\ _{\ }$&$4_{\ }$&$0_{4}$&$5_{\ }$\\
$+$$\ _{\ }$&$\ _{\ }$&$9_{1}$&$0_{\ }$\\
\cline{1-4}$\ _{\ }$&$4_{\ }$&$9_{\ }$&$5_{\ }$\\
\end{arithprob}
\hfil\begin{tabular}[t]{lr}Buswell&7.62\%\\Ainsworth&3.95\%\\Cox&3.54\%\\\end{tabular}\par\bigskip\bugitem{ignoreZX}{ignores-zero-multiplier} \nopagebreak The first multiplier is ignored when it's a zero, and no 
 zero is inserted in the answer row.\nopagebreak\par\nopagebreak\medskip\nopagebreak 
\begin{arithprob}{p{1em}p{1em}p{1em}}
$\ _{\ }$&$5_{\ }$&$3_{\ }$\\
$\times$$\ _{\ }$&$2_{\ }$&$0_{\ }$\\
\cline{1-3}$1_{\ }$&$0_{\ }$&$6_{\ }$\\
\end{arithprob}
\hfil\begin{tabular}[t]{lr}Cox&3.54\%\\Attisha&\\\end{tabular}\par\bigskip\bugitem{wrongleadingZ}{incorrect-number-of-annex-zeros} \nopagebreak An incorrect number of zeros are inserted into one of the
 answer rows.\nopagebreak\par\nopagebreak\medskip\nopagebreak 
\begin{arithprob}{p{1em}p{1em}p{1em}p{1em}p{1em}}
$\ _{\ }$&$\ _{\ }$&$4_{\ }$&$5_{\ }$&$6_{\ }$\\
$\ _{\ }$&$\times$$\ _{\ }$&$2_{\ }$&$5_{\ }$&$1_{\ }$\\
\cline{2-5}$\ _{\ }$&$\ _{\ }$&$4_{\ }$&$5_{\ }$&$6_{\ }$\\
$2_{\ }$&$2_{\ }$&$8_{\ }$&$0_{\ }$&$0_{\ }$\\
$9_{\ }$&$1_{\ }$&$2_{\ }$&$0_{\ }$&$0_{\ }$\\
\end{arithprob}
\hfil\begin{tabular}[t]{lr}Cox&6.19\%\\\end{tabular}\par\bigskip\bugitem{lastXlast}{last-digits-multiplied} \nopagebreak The last multiplicand is multiplied by the last multiplier, rather than
 multiply each multiplier by each multiplicand.  In the example, \x27=14,
 \x20=0+1=1, then \x53=15.\nopagebreak\par\nopagebreak\medskip\nopagebreak 
\begin{arithprob}{p{1em}p{1em}p{1em}p{1em}}
$\ _{\ }$&$5_{\ }$&$0_{\ }$&$7_{\ }$\\
$\times$$\ _{\ }$&$\ _{\ }$&$3_{\ }$&$2_{\ }$\\
\cline{1-4}$1_{\ }$&$5_{\ }$&$1_{1}$&$4_{\ }$\\
\end{arithprob}
\hfil\begin{tabular}[t]{lr}Cox&1.77\%\\\end{tabular}\par\bigskip\bugitem{notwotwo}{last-multiplication-skipped} \nopagebreak The second multiplicand is not multiplied by the second multiplier.\nopagebreak\par\nopagebreak\medskip\nopagebreak 
\begin{arithprob}{p{1em}p{1em}p{1em}}
$\ _{\ }$&$3_{\ }$&$2_{\ }$\\
$\times$$\ _{\ }$&$4_{\ }$&$1_{\ }$\\
\cline{1-3}$\ _{\ }$&$3_{\ }$&$2_{\ }$\\
$+$$\ _{\ }$&$8_{\ }$&$0_{\ }$\\
\cline{1-3}$1_{\ }$&$1_{\ }$&$2_{\ }$\\
\end{arithprob}
\hfil\begin{tabular}[t]{lr}Ainsworth&1.32\%\\\end{tabular}\par\bigskip\bugitem{XPbyC}{multiplied-product-by-carry} \nopagebreak The carry digit is multiplied by the product, rather than being
 added to it.  In this example, \x39=27, \x31=3, \x32=6.\nopagebreak\par\nopagebreak\medskip\nopagebreak 
\begin{arithprob}{p{1em}p{1em}p{1em}}
$\ _{\ }$&$1_{\ }$&$9_{\ }$\\
$\times$$\ _{\ }$&$\ _{\ }$&$3_{\ }$\\
\cline{1-3}$\ _{\ }$&$6_{2}$&$7_{\ }$\\
\end{arithprob}
\hfil\begin{tabular}[t]{lr}Cox&1.77\%\\Attisha&\\\end{tabular}\par\bigskip\bugitem{Xbyfirstdigit}{multiplies-all-by-first-multiplier} \nopagebreak The first multiplier is used to multiply all the other digits.  In this
 example, \x12=2, \x14=4, \x13=3.\nopagebreak\par\nopagebreak\medskip\nopagebreak 
\begin{arithprob}{p{1em}p{1em}p{1em}}
$\ _{\ }$&$4_{\ }$&$2_{\ }$\\
$\times$$\ _{\ }$&$3_{\ }$&$1_{\ }$\\
\cline{1-3}$3_{\ }$&$4_{\ }$&$2_{\ }$\\
\end{arithprob}
\hfil\begin{tabular}[t]{lr}Ainsworth&3.95\%\\\end{tabular}\par\bigskip\bugitem{XCbyremains}{multiplies-by-carry-over-blank} \nopagebreak When the multiplicand is over an empty cell, the subject multiplies
 by the carry digit.\nopagebreak\par\nopagebreak\medskip\nopagebreak 
\begin{arithprob}{p{1em}p{1em}p{1em}}
$\ _{\ }$&$7_{\ }$&$6_{\ }$\\
$\times$$\ _{\ }$&$\ _{\ }$&$4_{\ }$\\
\cline{1-3}$1_{\ }$&$4_{2}$&$4_{\ }$\\
\end{arithprob}
\hfil\begin{tabular}[t]{lr}Attisha&\\\end{tabular}\par\bigskip\bugitem{XCbymult}{multiplies-carry} \nopagebreak When there is a carry digit in the current column, it is used for
 multiplication instead of the multiplicand. I.e., \x84=32, \x34=12,
 and so on.\nopagebreak\par\nopagebreak\medskip\nopagebreak 
\begin{arithprob}{p{1em}p{1em}p{1em}p{1em}}
$\ _{\ }$&$3_{\ }$&$0_{\ }$&$8_{\ }$\\
$\times$$\ _{\ }$&$\ _{\ }$&$\ _{\ }$&$4_{\ }$\\
\cline{1-4}$\ _{\ }$&$4_{1}$&$2_{3}$&$2_{\ }$\\
\end{arithprob}
\hfil\begin{tabular}[t]{lr}Cox&5.31\%\\Attisha&\\\end{tabular}\par\bigskip\bugitem{Xlasttopwrites10}{multiplies-last-multiplicand-and-writes-10} \nopagebreak The only multiplication performed is to multiply the multiplier by the
 last multiplicand (\x36 in the example).  The product is written in
 the answer row, and ten is written after it.\nopagebreak\par\nopagebreak\medskip\nopagebreak 
\begin{arithprob}{p{1em}p{1em}p{1em}p{1em}}
$\ _{\ }$&$\ _{\ }$&$3_{\ }$&$0_{\ }$\\
$\ _{\ }$&$\times$$\ _{\ }$&$\ _{\ }$&$6_{\ }$\\
\cline{2-4}$1_{\ }$&$0_{\ }$&$1_{\ }$&$8_{\ }$\\
\end{arithprob}
\hfil\begin{tabular}[t]{lr}Cox&0.88\%\\\end{tabular}\par\bigskip\bugitem{Xmultiplicands}{multiplies-multiplicands} \nopagebreak The first multiplication is correct, but the subject then multiplies the
 multiplicands.  In this example, \x14=4, \x24=8.\nopagebreak\par\nopagebreak\medskip\nopagebreak 
\begin{arithprob}{p{1em}p{1em}p{1em}}
$\ _{\ }$&$2_{\ }$&$4_{\ }$\\
$\times$$\ _{\ }$&$3_{\ }$&$1_{\ }$\\
\cline{1-3}$\ _{\ }$&$8_{\ }$&$4_{\ }$\\
\end{arithprob}
\hfil\begin{tabular}[t]{lr}Ainsworth&1.32\%\\\end{tabular}\par\bigskip\bugitem{XpP}{multiplies-partial-product} \nopagebreak The partial product is multiplied, not added, with the bug
 \bug{XlikeA}.\nopagebreak\par\nopagebreak\medskip\nopagebreak 
\begin{arithprob}{p{1em}p{1em}p{1em}}
$\ _{\ }$&$3_{\ }$&$2_{\ }$\\
$\times$$\ _{\ }$&$2_{\ }$&$1_{\ }$\\
\cline{1-3}$\ _{\ }$&$3_{\ }$&$2_{\ }$\\
$6_{\ }$&$4_{\ }$&$0_{\ }$\\
\cline{1-3}$7_{1}$&$2_{\ }$&$0_{\ }$\\
\end{arithprob}
\hfil\begin{tabular}[t]{lr}Ainsworth&2.63\%\\\end{tabular}\par\bigskip\bugitem{XlikeA}{multiplies-using-addition-pattern} \nopagebreak Uses the addition pattern, but multiplies.\nopagebreak\par\nopagebreak\medskip\nopagebreak 
\begin{arithprob}{p{1em}p{1em}p{1em}p{1em}}
$\ _{\ }$&$5_{\ }$&$2_{\ }$&$4_{\ }$\\
$\times$$\ _{\ }$&$7_{\ }$&$3_{\ }$&$1_{\ }$\\
\cline{1-4}$3_{\ }$&$5_{\ }$&$6_{\ }$&$4_{\ }$\\
\end{arithprob}
\hfil\begin{tabular}[t]{lr}Ainsworth&11.84\%\\Cox&4.42\%\\Attisha&\\\end{tabular}\par\bigskip\bugitem{XbyCwhenZ}{multiply-by-carry-when-zero} \nopagebreak When the multiplicand is zero, the subject prefers to multiply by the
 carry digit.\nopagebreak\par\nopagebreak\medskip\nopagebreak 
\begin{arithprob}{p{1em}p{1em}p{1em}p{1em}p{1em}}
$\ _{\ }$&$\ _{\ }$&$4_{\ }$&$0_{\ }$&$6_{\ }$\\
$\ _{\ }$&$\times$$\ _{\ }$&$\ _{\ }$&$7_{\ }$&$3_{\ }$\\
\cline{2-5}$\ _{\ }$&$1_{\ }$&$2_{\ }$&$3_{1}$&$8_{\ }$\\
$3_{\ }$&$0_{2}$&$8_{4}$&$2_{\ }$&$0_{\ }$\\
\end{arithprob}
\hfil\begin{tabular}[t]{lr}Cox&1.77\%\\Attisha&\\\end{tabular}\par\bigskip\bugitem{nXZEn}{N$\times$0=N} \nopagebreak When N is multiplied by zero, N is the answer.\nopagebreak\par\nopagebreak\medskip\nopagebreak 
\begin{arithprob}{p{1em}p{1em}p{1em}p{1em}}
$\ _{\ }$&$3_{\ }$&$0_{\ }$&$2_{\ }$\\
$\times$$\ _{\ }$&$\ _{\ }$&$\ _{\ }$&$3_{\ }$\\
\cline{1-4}$\ _{\ }$&$9_{\ }$&$3_{\ }$&$6_{\ }$\\
\end{arithprob}
\hfil\begin{tabular}[t]{lr}Buswell&23.44\%\\Ainsworth&21.05\%\\Cox&12.39\%\\Attisha&\\\end{tabular}\par\bigskip\bugitem{nolead3}{no-annexing-in-third} \nopagebreak No zeros were inserted for the third answer row.\nopagebreak\par\nopagebreak\medskip\nopagebreak 
\begin{tabular}[t]{lr}Cox&2.65\%\\\end{tabular}\par\bigskip\bugitem{pPconf}{partial-product-confusion} \nopagebreak A general, combination error in which the subject had difficulty when
 the problem had two or more multipliers.  In the first example, \x45=20,
 \x24=8+2=10, \x21=2+1=3.  In the second example, the second and third
 products are written on the same answer row.\nopagebreak\par\nopagebreak\medskip\nopagebreak 
\begin{arithprob}{p{1em}p{1em}p{1em}}
$1_{\ }$&$4_{\ }$&$4_{\ }$\\
$\ _{\ }$&$2_{\ }$&$5_{\ }$\\
\cline{1-3}$3_{1}$&$0_{2}$&$0_{\ }$\\
\end{arithprob}
\hfil\begin{arithprob}{p{1em}p{1em}p{1em}p{1em}p{1em}p{1em}p{1em}p{1em}p{1em}}
$\ _{\ }$&$\ _{\ }$&$\ _{\ }$&$\ _{\ }$&$\ _{\ }$&$\ _{\ }$&$5_{\ }$&$1_{\ }$&$2_{\ }$\\
$\ _{\ }$&$\ _{\ }$&$\ _{\ }$&$\ _{\ }$&$\ _{\ }$&$\times$$\ _{\ }$&$\ _{\ }$&$2_{\ }$&$5_{\ }$\\
\cline{6-9}$\ _{\ }$&$\ _{\ }$&$\ _{\ }$&$\ _{\ }$&$\ _{\ }$&$\ _{\ }$&$5_{\ }$&$1_{\ }$&$2_{\ }$\\
$+$$\ _{\ }$&$1_{\ }$&$0_{\ }$&$2_{\ }$&$4_{\ }$&$1_{\ }$&$5_{\ }$&$3_{\ }$&$6_{\ }$\\
\cline{1-9}\end{arithprob}
\hfil\begin{tabular}[t]{lr}Buswell&6.25\%\\\end{tabular}\par\bigskip\bugitem{incorrectApP}{partial-product-incorrectly-summed} \nopagebreak The addition of the partial product is incorrect.  Cox apparently
 used this category to cover a number of addition bugs.\nopagebreak\par\nopagebreak\medskip\nopagebreak 
\begin{arithprob}{p{1em}p{1em}p{1em}p{1em}p{1em}}
$\ _{\ }$&$\ _{\ }$&$\ _{\ }$&$5_{\ }$&$3_{\ }$\\
$\ _{\ }$&$\ _{\ }$&$\times$$\ _{\ }$&$7_{\ }$&$4_{\ }$\\
\cline{3-5}$\ _{\ }$&$\ _{\ }$&$2_{\ }$&$1_{\ }$&$2_{\ }$\\
$+$$\ _{\ }$&$3_{\ }$&$7_{\ }$&$1_{\ }$&$0_{\ }$\\
\cline{1-5}$\ _{\ }$&$4_{\ }$&$4_{\ }$&$2_{\ }$&$2_{\ }$\\
\end{arithprob}
\hfil\begin{tabular}[t]{lr}Cox&4.42\%\\\end{tabular}\par\bigskip\bugitem{revpP}{partial-product-reversed} \nopagebreak The order of the digits is reversed in  the partial product. In the
 example, the ``219'' should be ``912''.\nopagebreak\par\nopagebreak\medskip\nopagebreak 
\begin{arithprob}{p{1em}p{1em}p{1em}p{1em}}
$\ _{\ }$&$4_{\ }$&$5_{\ }$&$6_{\ }$\\
$\times$$\ _{\ }$&$2_{\ }$&$5_{\ }$&$1_{\ }$\\
\cline{1-4}$\ _{\ }$&$4_{\ }$&$5_{\ }$&$6_{\ }$\\
$2_{\ }$&$2_{\ }$&$8_{\ }$&$0_{\ }$\\
$\ _{\ }$&$2_{\ }$&$1_{\ }$&$9_{\ }$\\
\end{arithprob}
\hfil\begin{tabular}[t]{lr}Buswell&1.17\%\\Cox&0.88\%\\\end{tabular}\par\bigskip\bugitem{QafterfirstX}{quits-after-first-multiplication} \nopagebreak Only the first multiplication is completed.\nopagebreak\par\nopagebreak\medskip\nopagebreak 
\begin{arithprob}{p{1em}p{1em}p{1em}p{1em}}
$\ _{\ }$&$2_{\ }$&$4_{\ }$&$7_{\ }$\\
$\times$$\ _{\ }$&$\ _{\ }$&$\ _{\ }$&$4_{\ }$\\
\cline{1-4}$\ _{\ }$&$\ _{\ }$&$2_{\ }$&$8_{\ }$\\
\end{arithprob}
\hfil\begin{tabular}[t]{lr}Attisha&\\\end{tabular}\par\bigskip\bugitem{Qafterfirstbottom}{quits-after-first-multiplier} \nopagebreak Only the first multiplier is used.\nopagebreak\par\nopagebreak\medskip\nopagebreak 
\begin{arithprob}{p{1em}p{1em}p{1em}p{1em}}
$\ _{\ }$&$3_{\ }$&$4_{\ }$&$6_{\ }$\\
$\times$$\ _{\ }$&$\ _{\ }$&$2_{\ }$&$8_{\ }$\\
\cline{1-4}$2_{\ }$&$7_{3}$&$6_{4}$&$8_{\ }$\\
\end{arithprob}
\hfil\begin{tabular}[t]{lr}Buswell&10.16\%\\Ainsworth&2.63\%\\Attisha&\\\end{tabular}\par\bigskip\bugitem{Qat100s}{quits-at-100s} \nopagebreak The subject quits multiplying after processing the tens column.\nopagebreak\par\nopagebreak\medskip\nopagebreak 
\begin{arithprob}{p{1em}p{1em}p{1em}p{1em}p{1em}}
$\ _{\ }$&$\ _{\ }$&$2_{\ }$&$2_{\ }$&$4_{\ }$\\
$\ _{\ }$&$\times$$\ _{\ }$&$1_{\ }$&$1_{\ }$&$8_{\ }$\\
\cline{2-5}$\ _{\ }$&$1_{\ }$&$7_{\ }$&$9_{\ }$&$2_{\ }$\\
$\ _{\ }$&$2_{\ }$&$2_{\ }$&$4_{\ }$&$0_{\ }$\\
\end{arithprob}
\hfil\begin{tabular}[t]{lr}Cox&0.88\%\\\end{tabular}\par\bigskip\bugitem{repX}{repeated-multiplication} \nopagebreak A multiplication was repeated.\nopagebreak\par\nopagebreak\medskip\nopagebreak 
\begin{arithprob}{p{1em}p{1em}}
$\ _{\ }$&$4_{\ }$\\
$\times$$\ _{\ }$&$2_{\ }$\\
\cline{1-2}$8_{\ }$&$8_{\ }$\\
\end{arithprob}
\hfil\begin{tabular}[t]{lr}Buswell&0.59\%\\\end{tabular}\par\bigskip\bugitem{skipsZmultiplicand}{skips-zero-multiplicand} \nopagebreak When the multiplicand contains a zero, the multiplication is skipped
 and the reminding digits of the multiplicand are multiplied by the
 multiplier directly under the zero.  In the example, \x29=18, \x85=40.\nopagebreak\par\nopagebreak\medskip\nopagebreak 
\begin{arithprob}{p{1em}p{1em}p{1em}p{1em}}
$\ _{\ }$&$8_{\ }$&$0_{\ }$&$9_{\ }$\\
$\times$$\ _{\ }$&$\ _{\ }$&$5_{\ }$&$2_{\ }$\\
\cline{1-4}$4_{\ }$&$0_{\ }$&$1_{\ }$&$8_{\ }$\\
\end{arithprob}
\hfil\begin{tabular}[t]{lr}Attisha&\\\end{tabular}\par\bigskip\bugitem{Zin100s}{spurious-zero-in-100s} \nopagebreak A zero is inserted in the hundreds column for no apparent reason.\nopagebreak\par\nopagebreak\medskip\nopagebreak 
\begin{arithprob}{p{1em}p{1em}p{1em}p{1em}p{1em}}
$\ _{\ }$&$\ _{\ }$&$9_{\ }$&$0_{\ }$&$5_{\ }$\\
$\ _{\ }$&$\times$$\ _{\ }$&$\ _{\ }$&$4_{\ }$&$6_{\ }$\\
\cline{2-5}$5_{\ }$&$4_{\ }$&$0_{\ }$&$3_{\ }$&$0_{\ }$\\
$3_{\ }$&$6_{\ }$&$0_{\ }$&$2_{\ }$&$0_{\ }$\\
\end{arithprob}
\hfil\begin{tabular}[t]{lr}Cox&0.88\%\\\end{tabular}\par\bigskip\bugitem{SpP}{subtracts-partial-product} \nopagebreak The subject subtracts the partial product rather than adding.  In this
 example the subject also subtracts the smaller number from the larger.\nopagebreak\par\nopagebreak\medskip\nopagebreak 
\begin{arithprob}{p{1em}p{1em}p{1em}p{1em}}
$\ _{\ }$&$\ _{\ }$&$5_{\ }$&$3_{\ }$\\
$\ _{\ }$&$\times$$\ _{\ }$&$7_{\ }$&$4_{\ }$\\
\cline{2-4}$\ _{\ }$&$2_{\ }$&$1_{\ }$&$2_{\ }$\\
$3_{\ }$&$7_{\ }$&$1_{\ }$&$0_{\ }$\\
\cline{1-4}$3_{\ }$&$5_{\ }$&$0_{\ }$&$2_{\ }$\\
\end{arithprob}
\hfil\begin{tabular}[t]{lr}Cox&0.88\%\\Attisha&\\\end{tabular}\par\bigskip\bugitem{manyZ}{too-many-annex-zeros} \nopagebreak Too many zeros are inserted into the answer row when
 multiplying by a multiple of ten.\nopagebreak\par\nopagebreak\medskip\nopagebreak 
\begin{arithprob}{p{1em}p{1em}p{1em}p{1em}p{1em}p{1em}p{1em}}
$\ _{\ }$&$\ _{\ }$&$\ _{\ }$&$\ _{\ }$&$5_{\ }$&$5_{\ }$&$3_{\ }$\\
$\ _{\ }$&$\ _{\ }$&$\ _{\ }$&$\times$$\ _{\ }$&$\ _{\ }$&$2_{\ }$&$0_{\ }$\\
\cline{4-7}$1_{\ }$&$1_{\ }$&$0_{\ }$&$6_{\ }$&$0_{\ }$&$0_{\ }$&$0_{\ }$\\
\end{arithprob}
\hfil\begin{tabular}[t]{lr}Cox&0.88\%\\Attisha&\\\end{tabular}\par\bigskip\bugitem{weird}{weird-order} \nopagebreak The digits are multiplied in a strange order.  In this example, the
 order is: \x41=4, \x21=2, \x23=6, \x24=8.\nopagebreak\par\nopagebreak\medskip\nopagebreak 
\begin{arithprob}{p{1em}p{1em}p{1em}p{1em}}
$\ _{\ }$&$\ _{\ }$&$1_{\ }$&$3_{\ }$\\
$\ _{\ }$&$\times$$\ _{\ }$&$2_{\ }$&$4_{\ }$\\
\cline{2-4}$8_{\ }$&$6_{\ }$&$2_{\ }$&$4_{\ }$\\
\end{arithprob}
\hfil\begin{tabular}[t]{lr}Ainsworth&1.32\%\\\end{tabular}\par\bigskip\bugitem{workslr}{works-left-to-right} \nopagebreak The subject starts at the left, adding carries to the right. In the
 example, \x53=15, \x23=6+1=7.\nopagebreak\par\nopagebreak\medskip\nopagebreak 
\begin{arithprob}{p{1em}p{1em}p{1em}}
$\ _{\ }$&$5_{\ }$&$2_{\ }$\\
$\times$$\ _{\ }$&$\ _{\ }$&$3_{\ }$\\
\cline{1-3}$\ _{\ }$&$5_{\ }$&$7_{1}$\\
\end{arithprob}
\hfil\begin{tabular}[t]{lr}Buswell&0.2\%\\\end{tabular}\par\bigskip\bugitem{firstZ}{zero-in-first-row} \nopagebreak A zero is inserted at the start of the first row.  Subsequent rows have
 the correct number of zeros.\nopagebreak\par\nopagebreak\medskip\nopagebreak 
\begin{arithprob}{p{1em}p{1em}p{1em}p{1em}p{1em}p{1em}}
$\ _{\ }$&$\ _{\ }$&$\ _{\ }$&$4_{\ }$&$3_{\ }$&$6_{\ }$\\
$\ _{\ }$&$\ _{\ }$&$\times$$\ _{\ }$&$\ _{\ }$&$5_{\ }$&$1_{\ }$\\
\cline{3-6}$\ _{\ }$&$\ _{\ }$&$4_{\ }$&$3_{\ }$&$6_{\ }$&$0_{\ }$\\
$+$$\ _{\ }$&$2_{\ }$&$1_{1}$&$8_{3}$&$0_{\ }$&$0_{\ }$\\
\cline{1-6}$\ _{\ }$&$2_{\ }$&$6_{1}$&$1_{\ }$&$6_{\ }$&$0_{\ }$\\
\end{arithprob}
\hfil\begin{tabular}[t]{lr}Cox&0.88\%\\Attisha&\\\end{tabular}\par\bigskip