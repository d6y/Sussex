\documentstyle[mya4,11pt,richardd,psychbib2]{article}
\pagestyle{empty}
\begin{document}
\bibliographystyle{psychbib}
\begin{flushleft}
{\large\bf Children's Simple Arithmetic: A Revisionist
Approach}\medskip\\
\rm\small Richard Dallaway $<$richardd@cogs.susx.ac.uk$>$\\
Monday 22 April 1991
\end{flushleft}
\parindent=0pt\parskip=1mm

Children's simple arithmetic skills \cite*{mindbugs,mcclcogn} is
an excellent domain for applying {\em revisionist connectionism}
\cite*{tourbolt,hintmapp,ptc}.  This approach considers symbolic AI
to be a faithful model of some aspects of cognition.  However, it is not
clear what notions like ``symbolic'' and ``manipulation''
actually entail. The hope is that connectionism will provide
insights into these questions.

By developing a model of human arithmetic skills (subtraction, long
multiplication, etc.), problems involved in building hybrid systems
have to be addressed: the nature of general rules and variables in
connectionism \cite*{micro,kirswhen}; how networks can learn and execute
sequential tasks \cite*{elmafind}; knowledge proceduralization and
redescription \cite*{innards}; the relation to production systems
\cite*{andearch}.

\bibliography{include,arith,cogs,pdp,prodsys,subsymbols}
\end{document}
